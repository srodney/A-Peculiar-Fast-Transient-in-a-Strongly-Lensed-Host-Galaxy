\begin{abstract}
Two unusual transient events were observed by the Hubble Space
Telescope in 2014, appearing in a galaxy at z=1.0054$\pm$0.0002 that
is strongly lensed by the foreground galaxy cluster \macs0416. These
transients---collectively nicknamed ``Spock''--- were faster and
fainter than any supernova, but significantly more luminous than a
classical nova: they reached peak luminosities of $\sim10^{41}$ erg
s$^{-1}$ in only $\lesssim$5 rest-frame days, then faded below
detectability in roughly the same timespan. Lens models of this
cluster suggest that it is entirely plausible that the two events are
{\it spatially} coincident on the source plane, but very unlikely that
they were also {\it temporally} coincident.  We find that Spock can
plausibly be classified as a recurrent nova, a luminous blue variable,
or a stellar caustic crossing event---but all three explanations
require either an extreme astrophysical source or highly unusual
gravitational lensing.
\end{abstract}
  
  
