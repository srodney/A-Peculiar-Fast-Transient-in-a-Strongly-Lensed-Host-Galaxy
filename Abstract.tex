\begin{abstract}
Two unusual transient events were observed by the Hubble Space
Telescope in 2014, appearing in a galaxy at $z=1.0054\pm0.0002$ that
is gravitationally lensed by the galaxy cluster \macs0416. These
transients---designated \spockone and \spocktwo and
collectively nicknamed ``Spock''---were faster and
fainter than any supernova, but significantly more luminous than a
classical nova. They reached peak luminosities of $\sim10^{41}$ erg
s$^{-1}$ (M$_{AB}<-14$) in $\lesssim$5 rest-frame days, then faded
below detectability in roughly the same time span.  Models of the
cluster lens suggest that these events may be {\it spatially}
coincident at the source plane, but are most likely not {\it
  temporally} coincident.  We find that \spock can be explained
as a luminous blue variable, a recurrent nova, or a pair of stellar
microlensing events.  To distinguish between these hypotheses will
require a clarification of the positions of nearby critical curves,
along with high-cadence monitoring of the field that could detect new
transient episodes in the host galaxy.
\end{abstract}
  
  
