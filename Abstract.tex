\begin{abstract}
In January and August of 2014, two unusual transient events were
observed in a strongly lensed galaxy at z=1.0054$\pm$0.0002.
Discovered by the FrontierSN team in Hubble Space Telescope (HST)
observations from the Hubble Frontier Fields (HFF) program, these
events are designated \spockone and \spocktwo, and collectively
nicknamed ``Spock''.  Both transient episodes were faster and fainter
than any of the broad class of supernova-like transients.  They both
rose to a peak absolute optical/ultraviolet luminosity of $M\sim-14$
mag ($10^{41}$ erg s$^{-1}$) in only $\lesssim$5 rest-frame days, and
then faded away below detectability in roughly the same amount of
time.  These events appeared in two adjacent arcs of a strongly lensed
galaxy that is multiply-imaged into at least 3 distinct images by the
gravitational potential of the galaxy cluster \MACS0416 (z=0.396).
Using five independent lens models of this cluster, we find it is
entirely plausible that the two events are {\it spatially} coincident
on the source plane, but very unlikely that they were also {\it
  temporally} coincident.  We compare these events to existing
categories of astrophysical transients and find that none of them can
readily account for all characteristics of the \spock events. The
light curves could be plausibly explained as optical/UV emission from
a neutron star merger (a kilonova), a white dwarf He shell explosion
(a .Ia supernova), eruptive episodes from a luminous blue variable
(LBV), or H explosions from an extremely luminous nova. Among these,
the nova model is the least disfavored, as it allows for a rapid
recurrence period with little or no intervening variability.  This
model would imply that the \spock system has the fastest known
recurrence timescale of any nova (3 to 5 months) and that \spock is
about 2 orders of magnitude more luminous than an average nova.  This
then suggests that the \spock system's primary star is a white dwarf
very close to the Chandrasekhar mass limit, and that it is drawing
mass from the secondary star at an extremely efficient rate
($>10^{-7}$ \Msun yr$^{-1}$), making it a potential Type Ia Supernova
progenitor candidate.  We conclude with suggestions for modeling
efforts and observational tests that could help to clarify the nature
of this unusual transient.
\end{abstract}
  
  
