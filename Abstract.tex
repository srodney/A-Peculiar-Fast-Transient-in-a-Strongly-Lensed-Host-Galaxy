\begin{abstract}
Two unusual transient events were observed by the Hubble Space
Telescope in 2014, appearing in a galaxy at $z=1.0054\pm0.0002$ that
is gravitationally lensed by the galaxy cluster \macs0416. These
transients---collectively nicknamed ``Spock''---were faster and
fainter than any supernova, but significantly more luminous than a
classical nova: they reached peak luminosities of $\sim10^{41}$ erg
s$^{-1}$ (M$_{AB}<-14$) in $\lesssim$5 rest-frame days, then faded below
detectability in roughly the same time span. Lens models of the
foreground cluster suggest that it is entirely plausible that the two
events are {\it spatially} coincident at the source plane, but very
unlikely that they were also {\it temporally} coincident.  We find
that {\it Spock} can be explained as a luminous blue variable, a
recurrent nova, or a pair of stellar microlensing events.
High-cadence monitoring of the field could help to distinguish between
these hypotheses by detecting new transient episodes at or near the
\spock locations. Improvements to the lens models are also needed to
clarify the position of the critical curves.
\end{abstract}
  
  
