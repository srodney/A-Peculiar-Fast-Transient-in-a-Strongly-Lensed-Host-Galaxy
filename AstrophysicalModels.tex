\section{Astrophysical Models}

To evaluate possible astrophysical explanations for the two \spock
events, let us first summarize the information we have:

\begin{enumerate}
\item{Two events appeared in separate images of a single galaxy,
  separated by $\sim$120 days in the rest frame (240 days in the
  observer frame). \label{itm:TwoEvents}}
\item{Lensing models predict a time delay between the two host images
  of $\pm50$ days or less in the observer frame.}
\item{Each event lasted for $<$15 rest-frame days\label{itm:FastLC}}
\item{After correcting for lensing magnification of $\mu\sim30$, both
  events reach a peak luminosity of $\sim10^{41}$ ergs s$^{-1}$
  \textcolor{red}{(NEED TO REFINE THIS NUMBER)}}
\item{\textcolor{red}{ADD IN X-RAY AND GAMMA RAY CONSTRAINTS}}
\end{enumerate}

To explain the observation of two transient episodes in different host
galaxy images (item 1), there are three possible
scenarios:

\begin{enumerate}[(A)]
\item{A single physical event (e.g., a stellar explosion) that
  appeared to us twice with an intervening time delay due to strong
  gravitational lensing.}
\item{Two separate events from distinct sources (e.g., two explosions
  from unrelated stars).}
\item{Two separate events from a single astrophysical source (e.g., a
  recurrent nova).}
\item{An intrinsically static source that appears transient due to time-varying magnification (e.g. a microlensing event).}
\end{enumerate}

The very rapid rise and fall of both light curves (observation 2)
is incompatible with any of the normal SN classes.  For both
thermonuclear white dwarf explosions (Type Ia) and massive star core
collapse explosions (Type Ib, Ic, and II) the optical light curve
after reaching peak brightness is primarily powered by the decay of
radioactive \NiFiftySix to \CoFiftySix, which leads to a minimum
decline rate of $\sim$0.1 mag day$^{-1}$.  For Type II SNe this decay
time can be extended into a plateau phase by the recession of the
photosphere via a recombination wave propagint inward through the
ionized H of the expanding outer shell.  In no case can a normal SN
powered by the \NiFiftySix decay chain exhibit the decline rate of
\TODO{measure the decline rate precisely} that has
been observed for \spock.

\todo{Discuss kilonova, .Ia and Fast optical transients}

\todo{Make a figure showing the light curves compared to those fast
  transient models}


\subsection{A Single Explosion}

Any single explosion model (Scenario A) requires that the observed
transient events must be coincident both in space and time.  All
lensing models evaluated here agree that spatial coincidence in the
source plane is entirely plausible, but coincidence in time is highly
unlikely. For this scenario to be tenable, we would have to assume
that a large systematic bias is similarly affecting all of the lens
models.  While we cannot rule this out, it makes the single explosion
model significantly less tenable. 

Under this scenario, the optical observations in January, 2014 and the
infrared observations in August, 2014 are from the same explosion. At
redshift $z=1$ these translate to rest-frame ultraviolet (UV) and
optical wavelengths, respectively, and in Figure \textcolor{red}{TBD}
we examine whether the observed colors in these bands are consistent
with the spectral energy distributions observed for known transients.
\todo{Line up the events at peak, measure colors, compare to SNe,
  novae, kilonovae, .Ia, and fast optical transients}.  We can not
compare the UV to optical colors, since that would require a
correction for differential magnification, and for this scenario we
have already assumed that the lens models are unreliable. \todo{Are
  the colors viable?}


\subsection{Two Separate Sources}

Invoking two separate and unrelated explosions (Scenario B) would
imply that the two events are {\it not} coincident at the source plane
in either space or time.  This is consistent with the lens models,
which makes this scenario initially more attractive.  However,
removing the spatial coincidence is problematic.  All five lens models
indicate that the locations of the two events are within \TODO{How
  many arcsec?  how many pc?}.

\TODO{Evaluate the SED of this host galaxy pixel by pixel to determine
  whether the spock locations are consistent with being the same
  location on the source plane. }

Having two unrelated explosions occurring in the same
year within such a small physical area would be plausible if the
explosions were from a very common source, but the rapid light curve
already rules out all normal categories of supernova explosions.
Since we have not observed any similar fast transients in any other
galaxy throughout the Frontier Fields survey, to accept this scenario
we would have to conclude that the \spock host galaxy is a very
unusual physical environment that provided uniquely fertile ground for
such transients.

\TODO{Evaluate the star formation and metallicity of this host galaxy
  to determine whether the host is very unique.}

Nevertheless, let us accept the premise that this host has generated
two separate rare explosions in the same year, and consider what rare
explosion categories could be separately consistent with the observed
light curves.

\TODO{Evaluate kilonova, .Ia and Fast Optical transient light curves
  compared to the Spock light curves}

\subsection{Two Events from the Same Source}

The final option is to allow for two separate explosive events powered
by the same astrophysical source (Scenario C).  In this case the
observed events must be spatially coincident but not coincident in
time, which is fully consistent with all lens model constraints. The
two \spock events would each have been repeated (either before or
after we observed it) at the other position due to the gravitational
lensing time delay.  This hypothesis supposes that those
``gravitational echoes'' were simply missed, as they landed in one of
the long periods without HST observations on this field.

Adopting this hypothesis immediately rules out any catastrophic
explosive events---such as a supernova or neutron star merger---in
which the progenitor system is completely destroyed or disrupted. This
scenario also does not admit any of the category of {\it variable}
sources (e.g. Cepheids, RR Lyrae, or Mira variables) that exhibit
periodic changes in flux due to pulsations of the stellar photosphere
but do not have sharp, isolated transient episodes.

There are two broad categories of astrophysical sources that can show
recurrent explosive events that would fit this scenario.  The first is
active galactic nuclei (AGN), in which transient episodes can be
driven by clumps of matter falling onto the accretion disk of a
supermassive black hole.  The second is a recurrent stellar explosion,
such as a recurrent nova (RN) or luminous blue variable (LBV) star.
In a RN system a white dwarf star accretes matter from a close binary
companion and experiences a surface explosion that leaves the system
intact to restart the cycle.  The LBV stars are very massive evolved
stars ($\sim$10-100 \Msun) that exhibit occasional outbursts or
eruptions associated with significant mass loss episodes -- although
the exact physical mechanism for these events remains unclear.

The AGN hypothesis is disfavored principally due to the quiescence of
the \spock sources between episodes. In addition to stochastic and
brief transient events, most AGN typically also exhibit slower
variation on much longer timescales, which is not observed at either
of the \spock locations. Furthermore, the spectrum of the \spock host
galaxy shows none of the broad emission lines that are often (though
not always) observed in AGN.  Finally, an AGN would necessarily be
located at the center of the host galaxy. The severe distortion of the
host galaxy images makes it impossible to identify the location of the
host center from the galaxy morphology, and any spatial reconstruction
from the lens models is not precise enough for a useful test.
However, since gravitational lensing is achromatic, if the \spock
positions are coincident with the host galaxy center, then the color
of the galaxy at each \spock location in images 11.1 and 11.2 should
be consistent with the color at the center of the less distorted image
11.3.  Figure~\ref{fig:HostGalaxyColor} shows that the rest-frame U-V
color for both \spockone and \spocktwo is in the range
\textcolor{red}{CHECK THIS:} $0.65\pm0.15$, while the center of the
galaxy in image 11.3 has $U-B=0.35\pm0.15$. Although by no means
definitive, this suggests that the \spock events were not located at
the physical center of the host galaxy.

