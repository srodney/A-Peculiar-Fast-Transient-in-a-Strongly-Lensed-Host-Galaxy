\section{Classification}
\label{sec:Classification}

For transients that do not easily fall within familiar categories, a
useful starting point for classification is to examine the object in
the phase space of peak luminosity versus decline time \citep[see,
  e.g.,][]{Kasliwal:2010}.  To infer the luminosity and decline time
for each \spock event, we combine the linear fits to the light curves
(shown in Figure~\ref{fig:LinearLightCurveFits}) with the predicted
range of lensing magnifications
(Figure~\ref{fig:LensModelContours}. For any assumed value for the
time of peak brightness, the light curve fits give us an estimate of
the ``observed'' peak magnitude and a corresponding rise-time and
decline-time measurement.  We then convert this extrapolated peak
magnitude to a luminosity (e.g., $\nu L_\nu$ in erg s$^{-1}$) by
first correcting for the luminosity distance assuming a standard \LCDM
cosmology, and then accounting for an assumed lensing magnification,
$\mu$.  At the end of all this, we have a grid of possible peak
luminosities for each event as a function of magnification and time of
peak (or, equivalently, the decline time).

Figures~\ref{fig:PeakLuminosityDeclineTimeWide} and
\ref{fig:PeakLuminosityDeclineTime} show the resulting
constraints on the peak luminosity and the decline time, which we
quantify as $t_2$, the time over which the transient declines by 2
magnitudes.  Shaded green and red bands represent the \spockone and
\spocktwo events, respectively, and in each case they incorporate the
allowed range for time of peak (see
Figure~\ref{fig:LinearLightCurveFits}) and the lensing magnification
($10<\mu<100$) as reported in Table~\ref{tab:LensModelPredictions}.
The two events are largely consistent with each other, and if both
events are representative of a single system (or a homogeneous class)
then the most likely peak luminosity and decline time (the region with
the most overlap) would be $L_{\rm pk}\sim10^{41}$ ergs/s and
$t_3\sim1.8$ days.

%In Figure~\ref{fig:PeakLuminosityDeclineTimeWide} we also demarcate
%regions of the luminosity--decline time phase space occupied by known
%or theorized SN-like transients.  

\subsection{Supernova-like Transients}

The colored regions along the right side of
Figure~\ref{fig:PeakLuminosityDeclineTimeWide} mark the luminosity and
decline times for SNe and SN-like transients. This includes the
familiar luminosity-decline relation of Type Ia SNe
\citep{Phillips:1993} and the broad heterogeneous class of Core
Collapse SNe, as well as less well-understood classes such as
Superluminous SNe \citep{Gal-Yam:2012,Arcavi:2016}, Type Iax SNe
\citep{Foley:2013a}, fast optical transients \citep{Drout:2014}, Ca-rich SNe \citep{Filippenko:2003,Perets:2011,Kasliwal:2012}, and
Luminous Red Novae \citep[also called intermediate luminosity red
  transients;][]{Munari:2002,Kulkarni:2007,Kasliwal:2011b}.  The \spock
events are incompatible with all of these explosion categories,
owing to the very rapid rise and fall of both \spock light curves, and
their relatively low peak luminosities of $\sim10^{41}$ erg s$^{-1}$.

Dashed boxes in Figure~\ref{fig:PeakLuminosityDeclineTimeWide}
represent categories of ``SN-like'' stellar explosions that have been
theoretically predicted and extensively modeled, but for which very
few viable candidates have actually been observed. Both of these
categories come closer to matching the observed characteristics of the
two \spock events, so they warrant closer scrutiny.

\subsubsection{Kilonova}

Also called a ``macronova'' or ``mini-supernova,'' this is a theorized
optical transient that may be generated by the merger of a neutron
star (NS) binary. Such a NS+NS merger can drive a relativistic jet that may be
observed as a Gamma Ray Burst (GRB) and would emit gravitational
waves.  These may also be accompanied by a very rapid optical light
curve (the kilonova component) that is driven by the radioactive decay
of r-process elements in the ejecta \citep{Li:1998,Kulkarni:2005}.  To
date there are two cases of fast optical transients associated with
GRB events, which have been interpreted as possible kilonovae
\citep{Perley:2009,Tanvir:2013}.  The \spock transients fall within
the range of theoretically predicted peak luminosity and decline times
for kilonovae. However, the rise time for the \spockone event is at
least 5 days in the rest-frame, which is significantly longer than the
$<1$ day rise expected for a kilonova
\citep[e.g.][]{Metzger:2010,Barnes:2013,Kasen:2015}.  Furthermore,
both \spock events are either significantly fainter or faster than the
optical light curves for the two existing kilonova candidates.

\subsubsection{.Ia Supernova}

The dashed oval in Figure~\ref{fig:PeakLuminosityDeclineTimeWide}
represents the ``.Ia'' class of He shell explosions
\citep{Bildsten:2007}. These are theorized to arise from AM Canum
Venaticorum (AM CVn) systems, which are binary star systems
transferring He onto a C/O or O/Ne WD primary
\citep{Warner:1995, Nelemans:2005}.  \citet{Bildsten:2007} argued that
these systems can build up enough He on the surface of the WD to
trigger a thermonuclear runaway and possibly a detonation. A typical
AM CVn system could produce $\sim$10 He shell flashes over $\sim10^6$
yr, while the He mass transfer rate is slow enough to admit thermally
unstable burning in the WD's accreted He shell.  The final He shell
flash is the brightest, and is what we refer to as the .Ia event. This
last explosion may or may not lead to a detonation of the WD core
\citep[the double detonation scenario;][]{Nomoto:1982a, Nomoto:1982b,
  Woosley:1986, Woosley:1994}.

Theoretical .Ia models suggest that the light curves would be quite
bright, reaching a peak luminosity of $\sim10^{42}$ erg s$^{-1}$.
That is comparable to the brightness of a normal SN, but the .Ia light
curves would decline much more quickly.  After an initial short peak
(3-5 days) driven by the rapid radioactive decay of \isotope{48}{Cr}
and \isotope{52}{Fe} at the exterior of the ejecta, a secondary
decline phase kicks in, powered by the slower \isotope{56}{Ni} decay
chain \citep{Shen:2010}. The optical emission is expected to fade by 2
magnitudes after $\sim10$ days.  There have been a few viable .Ia
candidates presented in the literature \citep{Kasliwal:2010,
  Perets:2010, Poznanski:2010}, but we do not have enough objects to
empirically constrain .Ia light curve shapes.  Although the \spock
light curves were somewhat fainter and faster than the expectations
for a .Ia event, there is enough uncertainty about the diversity of
.Ia light curves that this model should not be dismissed on those
merits alone.

\subsubsection{The Recurrence Problem}

An additional challenge for applying any SN-like transient model to
explain the \spock events is the problem of the apparent
recurrence.  For all of these catastrophic stellar explosions we do
not expect to see repeated transient events: the kilonova progenitor
system is completely destroyed by the merger, and for the .Ia
explosions the principal observed transient event is the last
transient episode that system produces.  Even if we suppose that an AM
CVn system could produce repeated He shell flashes of similar
luminosity, the period of recurrence would be of order $10^5$ yr,
making these effectively non-recurrent sources.

Thus, the only way to reconcile these cataclysmic explosion models
with the two observed \spock events is to either (a) assert that the
two events are two images of the same explosion, appearing to us
separately only because of a gravitational lensing time delay
\citep[as was the case for the 5 images of SN
  Refsdal][]{Kelly:2015a,Kelly:2016}, or (b) invoke a highly
serendipitous occurrence of two unrelated peculiar explosions in the
same host galaxy in the same year.

To evaluate scenario (a), in which a lensing time delay causes the
appearance of two separate events, we must rely on the available lens
modeling. We have seen in Section~\ref{sec:LensingModels} that none of
the \macs0416 lens models predict an 8 month time delay between
appearances in image 11.1 and 11.2.  This is represented in
Figure~\ref{fig:SpockDelayPredictions}, where we have plotted the
light curves for the two transient events, along with shaded vertical
bars marking the time delay predictions of all models.
%The lens models are broadly consistent with each other, predicting
%that the lensing time delay between images 11.1 and 11.2 is on the
%order of $\pm$60 days, far short of the 238 day lag that was observed
%between \spockone\ and \spocktwo.
To accept this time-delayed single explosion explanation for \spock,
we would have to assume that a large systematic bias is similarly
affecting all of the lens models.  While we cannot rule out such a
bias, the consistency of the lens modeling makes this scenario less
tenable.

For the latter scenario of two unrelated explosions, it is difficult
to assess the likelihood of such an occurrence quantitatively, as
there are no measured rates of .Ia or kilonovae.  In a study of very
fast optical transients with the Pan-STARRS1 survey,
\citet{Berger:2013b} derived a limit of $\lesssim0.05$ Mpc$^{-3}$
yr$^{-1}$ for transients reaching $M\approx -14$ mag on a timescale of
$\sim$1 day.  This limit, though several orders of magnitude higher
than the constraints on novae or SNe, is sufficient to make it
exceedingly unlikely that two such transients would appear in the same
galaxy in a single year.  Furthermore, we have observed no other
transients with similar luminosities and light curve shapes in our
high-cadence surveys of 5 other Frontier Fields clusters. Indeed, all
other transients detected in the core Frontier Fields survey have been
fully consistent with normal SNe.  Thus, we have no evidence to
suggest that transients of this kind are much more common at $z\sim1$.


\subsection{Luminous Blue Variable}

The transient sources categorized as Luminous Blue Variables (LBVs)
are the result of eruptions or explosive episodes from massive stars
($>10$\Msun).  The class is exemplified by well-studied examples such
as P Cygni and $\eta$ Carinae (\etacar) in the Milky Way and S Doradus
(S-Dor) in the Large Magellanic Cloud \citep[for recent overviews of
  the LBV class, see][]{Smith:2011b, Kochanek:2012}.  Although the
association with massive stars is well established, this class is very
heterogeneous and there is currently a vigorous debate over the
precise nature of the progenitor pathway
\citep{Smith:2015,Humphreys:2016,Smith:2016}.  The ``Great Eruptions''
of such massive stars are sometimes labeled as ``SN impostors''
because these most prominent transient episodes can exhibit light
curves reminiscent of core collapse SNe, reaching peak absolute
magnitudes of $\sim$-7 to -16 mag in optical bands, and lasting for
tens to hundreds of days.  In some cases the LBV progenitor does
indeed culminate with a final true core collapse SN event
\citep[e.g.]{Mauerhan:2013, Tartaglia:2016}
%  Foley:2007,Pastorello:2007,Smith:2010b,GalYam:2009}.

Although most giant LBV eruptions have been observed to last much
longer than the \spock events \citep{Smith:2011b}, some LBVs have
exhibited repeated rapid outbursts that are broadly consistent with
the very fast \spock light curves. Because of this commonly seen
stochastic variability, the LBV scenario does not have any trouble
accounting for the \spock events as two separate episodes.

Two well-studied LBVs in particular provide a plausible match to the
observed \spock events.  The first is the transient ``SN 2009ip''
\citep{Maza:2009} which was later re-classified as an LBV as it showed
repeated brief transient episodes \citep[e.g.,][]{Miller:2009,
  Li:2009, Berger:2009, Drake:2010}. Pre-eruption HST imaging
demonstrated that the progenitor of SN 2009ip was likely a very high
mass star \citep[$\gtrsim50$ \Msun,][]{Smith:2010, Foley:2011}.
Remarkably, this star eventually did explode as a true SN event,
observed in 2012 \citep{Mauerhan:2013, Pastorello:2013, Prieto:2013}.

The second useful comparison object is NGC3432-LBV1 (also called SN
2000ch), which was first observed as a bright variable star
\citep{Papenkova:2000} and later definitively classified as an LBV
\citep{Wagner:2004}.  This event exhibited at least three significant
outbursts over 2-year period, which were observed in a concerted
monitoring campaign \citep{Pastorello:2010}.  The spectral
characteristics of this LBV suggest a similarity to Wolf-Rayet stars
\citep{Pastorello:2010} and the variation of the SED suggests
modulated dusty wind \citep{Wagner:2004, Kochanek:2012}. The observed
sequence of erratic transient episodes may also be indicative of
binary interactions similar to S-Dor \citep{Pastorello:2010,
  Smith:2011b}.


Figure~\ref{fig:LBVLightCurveComparison} presents a direct comparison
of the observed \spock light curves against the light curves of these
two rapid-eruption LBVs, SN 2009ip and NGC3432-LBV1. The brief
outbursts of these LBVs have been less finely sampled than the two
\spock events, but the available data show a wide variety of rise and
decline times, even for a single object over a relatively narrow time
window of a few months. For each of the rapid LBV outbursts shown in
Figure~\ref{fig:LBVLightCurveComparison} we have measured the peak
luminosity and the decline time, allowing these events to be plotted
in the $L_{\rm pk}$ vs. $t_2$ space of
Figure~\ref{fig:PeakLuminosityDeclineTime} (as orange diamonds).  All
of the rapid LBV eruptions of SN 2009ip and NGC3432-LBV1 provide only
upper limits on $t_2$, due to the relatively sparse photometric
sampling.  The observations of both \spock events are consistent with
the observed luminosities and decline times of the fastest and
brightest of rapid LBV outbursts. 

In addition to the relatively short and very bright giant eruptions
shown in Figure~\ref{fig:LBVLightCurveComparison}, most LBVs also
commonly exhibit a slower underlying variability that has not been
observed at the \spock locations. P Cygni and \etaCar, for example,
slowly rose and fell in brightness by $\sim$1 to 2 mag over a timespan
of several years before and after their historic giant eruptions.
Such variation has not been detected at the \spock locations, as can
be seen from the wide views of the \spock light curves in
Figure~\ref{fig:SpockDelayPredictions}. Nevertheless, given the broad
range of light curve behaviors seen in LBV events, we can not reject
this class as a possible explanation for the \spock system.

\todo{Measure this more quantitatively: forced photometry in drz (not
  diff) images at all epochs, estimate what would be the magnitude of
  a quiescent eta-Car-like star, and would we be able to detect a 1-2
  mag brightening over the span of the HFF campaign}



\subsection{Recurrent Nova}

Novae are represented in
Figure~\ref{fig:PeakLuminosityDeclineTimeWide} as a grey band, which
traces the maximum magnitude - rate of decline (MMRD) relation.  Nova
explosions occur in binary star systems in which the more massive star
is a white dwarf that accretes matter from its companion, which may be
a main sequence dwarf or evolved giant star overfilling its Roche
Lobe. The white dwarf builds up a dense layer of H-rich material on
its surface until the high pressure and temperature triggers nuclear
fusion, resulting in a surface explosion that causes the white dwarf
to brighten by several orders of magnitude, but does not completely
disrupt the star. In a recurrent nova (RN) system, the mass transfer
from the companion to the white dwarf restarts after the explosion, so
the cycle may begin again and repeat after a period of months or
years.

The seminal work of \citet{Zwicky:1936} and \citet{McLaughlin:1939}
first showed that more luminous novae within the Milky Way tend to
have more rapidly declining light curves, which is now the basis of
the maximum-magnitude versus rate-of-decline (MMRD) relationship. The
basic form of the MMRD relation has been theoretically attributed to a
dependence of the peak luminosity on the mass of the accreting white
dwarf \citep[e.g.][]{Livio:1992}.  Studies of extragalactic novae
reaching as far as the Virgo cluster have shown that the MMRD relation
is broadly applicable to all nova populations, though with significant
scatter
\citep[e.g.][]{Ciardullo:1990,DellaValle:1995,Ferrarese:2003,Shafter:2011}.
Amidst that scatter, there may also be sub-populations of novae that
deviate from the traditional MMRD form \citep{Kasliwal:2011a}, and
recurrent novae (RNe) in particular may be poorly represented by the
MMRD \citep{Shafter:2011,Hachisu:2015}

In Figure~\ref{fig:PeakLuminosityDeclineTimeWide}, the dark grey
region follows the empirical constraints on the MMRD from
\citet{DellaValle:1995}, and the wider light grey band allows for the
increased scatter about that relation that has been noted from more
extensive surveys of novae in the Milky Way \citep{Downes:2000}, M31
\citep{Shafter:2011} and elsewhere in the local group
\citep{Kasliwal:2011a}.  Nova outbursts can exhibit decline times from
$\sim$1 day to many months, so the timescale of the \spock light
curves can easily be accommodated by the nova scenario. However, the
peak luminosities inferred for the \spock events are larger than any
known novae, perhaps by as much as 2 orders of magnitude.

Figure~\ref{fig:PeakLuminosityDeclineTime} shows a narrower slice of
the same phase space as in
Figure~\ref{fig:PeakLuminosityDeclineTimeWide}, zooming in on the
``fast and faint'' region from the lower left corner.  The observed
constraints from the two published kilonova candidates are shown,
which provide only lower limits on the peak luminosity
\citep{Tanvir:2013}, or the decline timescale \citep{Perley:2009}.
Two .Ia candidates are also plotted, SN 2002bj \citep{Poznanski:2010}
and SN 2010X \citep{Kasliwal:2010}.  The sample of observed nova
outbursts (shown as solid points) demonstrates the observed scatter
about the MMRD relation.

One primary line of evidence supporting the nova hypothesis
comes from the \spock light curves. Many RN light curves are similar
in shape to the \spock episodes, exhibiting a sharp rise ($<10$ days
in the rest-frame) and a similarly rapid decline.
Figure~\ref{fig:RecurrentNovaLightCurveComparison} compares the \spock
light curves to template light curves from RNe within our galaxy and
in M31.  There are 10 known RNe in the Milky Way galaxy, and 7 of
these exhibit outbursts that decline rapidly, fading by 2 magnitudes
in less than 10 days \citep{Schaefer:2010}
% U Sco, V2487 Oph, V394 CrA, T CrB, RS Oph, V745  Sco, and V3890 Sgr.
The gray shaded region in
Figure~\ref{fig:RecurrentNovaLightCurveComparison} encompasses the V
band light curve templates for all 7 of these events, from
\citet{Schaefer:2010}.  The Andromeda galaxy (M31) also hosts at least
one RN with a rapidly declining light curve.  The 2014 eruption of
this well-studied nova, M31N 2008a-12, is shown as a solid black line
in Figure~\ref{fig:RecurrentNovaLightCurveComparison}, fading by 2
mags in less than 3 days.  This comparison demonstrates that the
sudden disappearance of both of the \spock transient events is fully
consistent with the eruptions of known RNe in the local universe.

The rise time of the \spock events is somewhat out of the ordinary for
nova outbursts.  In particular, for recurrent nova eruptions that
decline rapidly ($t_2<10$ days) they tend to also reach peak
brightness very quickly, on timescales $<1$ day
\citep{Schaefer:2010}. The 2014 eruption of the rapid-recurrence nova
M31N 2008a-12 reached maximum brightness in a little under 1 day
\citep{Darnley:2015}.  However, the rise time for nova eruptions is
poorly constrained, as rapid-cadence imaging is rarely secured until
after an initial detection near peak brightness.  Unlike the situation
with a kilonova light curve, there is no a priori physical expectation
for an especially rapid rise to peak in nova light curves.

Among the most luminous classical novae known, a similarly rapid
decline time is not unheard of.  For example, the bright nova
M31N-2007-11d had $t_2 = 9.5$ days \citep{Shafter:2009}.  The
extremely luminous nova SN 2010U had $t_2 = 3.5 \pm 0.3$
\citep{Czekala:2013}.  The nova L91 declined with $t_2 = 6 \pm 1$ days
\citep{DellaValle:1991, Schwarz:2001, Williams:1994, Schwarz:2001}.
The rise to maximum of L91 is also quite long, measure at least 4
days. \citet{Shafter:2009}.


Another reason to consider the RN model is that it provides a
natural explanation for having two separate explosions that are
coincident in space but not in time.  If \spock is a RN, then the two
observed episodes can be attributed to two distinct nova eruptions,
and the gravitational lensing time delay does not need to match the
observed 8 month separation between the January and August 2014
appearances.

Although {\it qualitatively} consistent with the 8-month separation,
the RN model is strained by a quantitative assessment of the
recurrence period. If \spock is indeed a RN at $z=1$, then the
recurrence timescale in the rest-frame is $120\pm30$ days ($3-5$
months), where the uncertainty accounts for the $1\sigma$ range of
modeled gravitational lensing time delays.  This would be a singularly
rapid recurrence period, significantly faster than all 11 RNe in our
own galaxy, which have recurrence timescales ranging from 15 years (RS
Oph) to 80 years (T CrB). For the 5 galactic RNe with a rapidly
declining outburst light curve (U Sco, V2487 Oph, V394 CrA, T CrB, and
V745 Sco), the median recurrence timescale is 21 years.  The fastest
measured recurrence timescale belongs to the Andromeda galaxy nova
M31N 2008a-12, which has exhibited a new outburst every year from
2009-2015
\citep{Tang:2014,Darnley:2014,Darnley:2015,Henze:2015,Henze:2015a}. Although
this M31 record-holder demonstrates that very rapid recurrence is
possible, classifying \spock as a RN would still require a very
extreme mass transfer rate to accommodate the $<1$ year recurrence.

Another major concern with the RN hypothesis for \spock is apparent
in Figure~\ref{fig:PeakLuminosityDeclineTime}, which shows that the
two \spock events are substantially brighter than all known novae --
perhaps by as much as 2 orders of magnitude.  One might attempt to
reconcile the \spock luminosity more comfortably with the nova class
by assuming a significant lensing magnification for one of the two
events. This would drive down the intrinsic luminosity, perhaps to
$\sim10^{40}$ erg s${-1}$, on the edge of the nova region.  However,
this assumption implicitly moves the lensing critical curve to be
closer to the \spock event in question.  That pulls the critical curve
away from the other \spock position, which makes that second event
{\it more inconsistent} with observed nova peak luminosities.  


\section{Non-explosive Astrophysical Transients}

There are several categories of astrophysical transients that are not
related to stellar explosions, and we find that these models cannot
accommodate the observations of the \spock transients.  We may first
dismiss any of the category of {\it periodic} sources (e.g. Cepheids,
RR Lyrae, or Mira variables) that exhibit regular changes in flux due
to pulsations of the stellar photosphere. These variable stars do not
exhibit sharp, isolated transient episodes that could match the \spock
light curve shapes.

We can also rule out active galactic nuclei (AGN), in which brief
transient episodes (a few days in duration) may be observed from X-ray
to infrared wavelengths \citep[e.g.][]{Gaskell:2003}.  The AGN
hypothesis for \spock is disfavored for three primary reasons:
%principally due to the quiescence of the
%\spock sources between the two observed episodes.
First, AGN that exhibit short-duration transient events also typically
exhibit slower variation on much longer timescales, which is not
observed at either of the \spock locations. Second, the spectrum
of the \spock host galaxy shows none of the broad emission lines that
are often (though not always) observed in AGN.  Third, an AGN would
necessarily be located at the center of the host galaxy.
%The severe distortion of the
%host galaxy images makes it impossible to identify the location of the
%host center in images 11.1 and 11.2 from the galaxy morphology. Any
%spatial reconstruction at the source plane from the lens models would
%not be not precise enough for a useful test.  However, since
%gravitational lensing is achromatic, if the \spock positions are
%coincident with the host galaxy center, then the color of the galaxy
%at each \spock location in images 11.1 and 11.2 should be consistent
%with the color at the center of the less distorted image 11.3.
In Section~\ref{sec:HostGalaxy}
we saw that there are minor differences in the host galaxy properties
(i.e. rest-frame U-V color and mean stellar age) from the \spockone
and \spocktwo locations to the center of the host galaxy at image 11.3
Although by no means definitive, this suggests that the \spock events
were not located at the physical center of the host galaxy, and
therefore are not related to an AGN. \todo{Update with MUSE results}

Stellar flares provide a third very common source for optical
transient events. Relatively mild stellar flares may be caused by
magnetic activity in the stellar atmosphere, and the brightest flare
events (so-called ``superflares'') may be generated by perturbations
to the stellar atmosphere via interactions from a disk, a binary
companion, or a planet.  In these circumstances the stars release a
{\em total} energy in the range of $10^{33}$ to $10^{38}$ erg over a
span of minutes to hours \citep{Balona:2012,Karoff:2016}. This falls
far short of the observed energy release from the \spock transients,
so we can also dismiss stellar flares as implausible for this source.

\subsection{Microlensing}

In the presence of strong gravitational lensing it is possible to
generate a transient event from lensing effects alone.  In this case
the background source has a steady luminosity but the relative motion
of the source, lens, and observer causes the magnification of that
source to change rapidly with time.

A commonly observed example is the microlensing of a bright background
source (a quasar) by a galaxy-scale lens \citep{Wambsganss:2001,
  Kochanek:2004}.  In this optically thick microlensing regime, the
lensing potential along the line of sight to the quasar is composed of
many stellar-mass objects.  Each compact object along the line of
sight generates a separate critical lensing curve, resulting in a
complex web of overlapping critical curves. As all of these lensing
stars are in motion relative to the background source, the web of
caustics will shift across the source position, leading to a
stochastic variability on timescales of months to years.  This
scenario is inconsistent with the observed data, as the two \spock
events were far too short in duration and did not exhibit the repeated
``flickering'' variation that would be expected from optically thick
microlensing.

A second possibility is through an isolated strong lensing event with
a rapid timescale, such as a background star crossing over a lensing
critical curve.  This corresponds to the optically thin microlensing
regime, and is similar to the ``local'' microlensing light curves
observed when stars within our galaxy or neighboring dwarf galaxies
pass behind a massive compact halo object \citep{Paczynski:1986,
  Alcock:1993, Aubourg:1993, Udalski:1993}.  In the case of a star
crossing the caustic of a smooth lensing potential, the amplification
of the source flux would increase (decrease) with a characteristic
$t^{-1/2}$ profile as it moves toward (away from) the caustic. This
slowly evolving light curve transitions to a very sharp decline (rise)
when the star has moved to the other side of the caustic
\citep{Schneider:1986, MiraldaEscude:1991}.  With a more complex lens
comprising many compact objects, the light curve would exhibit a
superposition of many such sharp peaks \citep{Lewis:1993}.

To generate an isolated microlensing event, the background source
would have to be the dominant source of luminosity in its environment,
meaning it must be a very bright O or B star with mass of order 10
\Msun.  Depending on its age, the size of such a star would range from
a few to a few dozen times the size of the sun.  The net relative
transverse velocity would be on the order of a few 100 km/s, which is
comparable to the orbital velocity of stars within a galaxy or
galaxies within a cluster.  In the case of a smooth cluster potential---the
%timescale
%$\tau$ for the light curve of such a caustic crossing event is
%dictated by the radius of the source, $R$, and the net transverse
%velocity, $v$, of the source across the caustic, as:
%
%\begin{equation}
%  \tau = 6\frac{R}{5\,\Rsun}\frac{300 {\rm km~ s}^{-1}}{v}~\rm{hr}
%\label{eqn:caustic_crossing_time}
%\end{equation}
%
%
%\noindent Thus, the
characteristic timescale of such an event would be on the order of
hours or days \citet{Chang:1979,Chang:1984,MiraldaEscude:1991}, which
is in the vicinity of the timescales observed for the \spock events.
However, if we apply this scenario to the \macs0416 field, we can not
plausibly generate two events with similar decay timescales at
distinct locations on the sky.  This is because a caustic-crossing
transient event must necessarily appear at the location of the lensing
critical curve, but in this case the critical curve most likely passes
between the two observed \spock locations. At best, a caustic crossing
could account for only one of the \spock events, not both.
