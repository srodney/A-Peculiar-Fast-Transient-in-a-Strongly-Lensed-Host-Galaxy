\subsection{Color Curves}\label{sec:ColorCurves}

At redshift $z=1$ the observed optical and infrared bands translate to
rest-frame ultraviolet (UV) and optical wavelengths, respectively.  To
derive rest-frame UV and optical colors from the observed photometry,
we start with the measured magnitude in a relatively blue band (F435W
and F606W for \spockone and F105W, F125W, F140W for \spocktwo).  We
then subtract the coeval magnitude for a matched red band (F814W for
\spockone, F125W or F160W for \spocktwo), derived from the linear fits
to those bands.  To adjust these to rest-frame filters, we apply K
corrections\citep{Hogg:2002}, which we compute by
defining a crude SED via linear interpolation between the observed
broad bands for each transient event at each epoch.  For consistency
with past published results, we include in each K correction a
transformation from AB to Vega-based magnitudes.  The resulting UV and
optical colors are plotted in Figure~\ref{fig:ColorCurves}.  Both
\spockone and \spocktwo show little or no color variation over the
period where color information is available.  This lack of color
evolution is compatible with all three of the primary hypotheses
advanced, as it is possible to have no discernible color evolution
from either an LBV or RN over this short time span, and microlensing
events inherently exhibit an unchanging color.
