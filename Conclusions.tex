\section{Conclusions}
\label{sec:Conclusions}

Although the observational evidence is limited,
\todo{summarize the observational constraints}

\todo{summarize the major uncertainties}

We have examined these two models in comparison to three primary
observational constraints: the light curve decline time (\t2), the
peak luminosity (\Lpk), and the recurrence timescale (\trec).

We have found that the RN model is physically untenable. The light
curve shape is consistent with RN systems in our own galaxy and M31,
but the peak luminosity and recurrence timescale are at (or beyond)
the limits of the known RN population.  Indeed, these latter two
observables are, at best, only barely compatible with the physical
limits of the RN model. The inferred peak luminosity of the \spock
events would require a primary white dwarf that is extraordinarily
close to the Chandresekhar limit.  The uniquely rapid recurrence
timescale would imply a mass transfer rate from the secondary star
that is remarkably fast ($>10{-7} \Msun$ yr${-1}$).  It is unclear
whether these two extremes are even physically compatible as a RN
system, as such a rapid mass transfer onto such a massive white dwarf
would likely result in stable nuclear burning at the surface, and
would therefore not lead to explosive burning episodes that could be
observed as rapid transient events.

Our preferred explanation for the \spock events is that we have
observed two distinct eruptive episodes from a massive LBV star. 
\todo{Reiterate the phenomenoligical consistency}

We have found that the \spock events are compatible with the observed
properties of known LBV eruptions from the local universe.
\todo{Reiterate the observational and physical consistency}
The very luminous and very fast \spock
transients may be driven by extreme mass eruption events, an extreme
form of stellar pulsation, or may be caused by a different mechanism
entirely.

These events will only add to the theoretical challenge of deriving a
physical model that accommodates both the great eruptions and the S
Dor-type variation of LBVs.

\todo{Close with a prescription for resolving the classification with
  regular monitoring of the field, and getting measurable time delays
  that could provide a rigorous test of the cluster mass models}

