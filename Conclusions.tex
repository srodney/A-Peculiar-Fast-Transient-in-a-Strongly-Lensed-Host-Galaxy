\section{Summary and Conclusions}
\label{sec:Conclusions}

The peculiar transients \spockone and \spocktwo behind \macs0416 have
presented an intriguing puzzle.  It is immediately apparent that these
rapidly evolving transients do not fit neatly within any of the common
categories of explosive or eruptive astrophysical sources. To classify
this unusual pair of events and interrogate their physical basis, we
have combined the constraints from five models of the lensing cluster,
high-resolution imaging and spectroscopy of the host galaxy, and
high-cadence \HST sampling of the two light curves.  Although this
observational evidence is limited by significant uncertainties, we
have derived three key observable parameters for the \spock system:
(1) both events reached a peak luminosity that was within an order of
magnitude of 10$^{41}$ erg s$^{-1}$ ($M_V=-14$); (2) the rise and
decline timescales for both events was $\sim$2--5 days in the rest
frame; (3) the two events are most likely spatially coincident but not
coincident in time---with a separation of 3-5 months in the rest frame.

The most significant uncertainty in the luminosity (item 1) comes from
the gravitational lens modeling.  A lensing critical curve of the
\macs0416 cluster passes within 3\arcsec of both \spock events, which
makes it very difficult to pin down a prediction for the lensing
magnification.  From our 5 independent lens models we derive a
plausible range of magnifications ranging from $\mu\sim$10 to 100. In
spite of this large uncertainty, our luminosity constraint is still
sufficient to rule out stellar flares on the low end, and most types
of supernovae on the high end.

A second major uncertainty affecting both our constraints on the peak
luminosity and on the rise and decline times of the light curves comes
from the absence of photometric sampling immediately around the peak
of each light curve.  The rapidly evolving transient episodes were not
completely captured by the high-cadence imaging sequence of the HFF
program, leaving gaps of 3 and 14 days for \spockone and \spocktwo,
respectively.  Nevertheless, using simplistic assumptions about the
light curve behavior, we can reliably constrain the time each event
took to decline by 2 magnitudes to $\t2<6$ days---significantly
faster than almost all categories of explosive stellar phenomena.

The luminosities and rapid light curves for both \spock events are
marginally compatible with two categories of theorized optical
transients: the kilonova and .Ia classes.  A key problem with these
models is that both are expected to be intrinsically quite rare, with
a rate that is orders of magnitude less than the rate for normal SN
events.  Either the merger of a NS binary (leading to a kilonova) or
the final shell explosion of an AM CVn system (causing a .Ia event)
would be a terminal event---there would be no opportunity for a second
transient event.  This conflicts with the third of our primary
observational constraints, that the two \spock transients were
probably caused by two separate events at the source plane.

That inference that the two events are from a single {\em recurrent}
source is derived from analysis of our diverse set of lens models.
All agree that the gravitational lensing time delay between the two
positions of the \spock events is far shorter than the observed time
difference of $\sim$8 months.  This conclusion could be assailed by
moving the location of the lensing critical curve, so that it is
significantly closer to one or the other of the two events.  This
would drive up the difference in the lensing potential between the two
\spock positions, and could lengthen the lensing time delay, which is
directly proportional to the lensing potential difference.  However,
this would also substantially increase the inferred lensing
magnification of one event, and simultaneously decrease $\mu$ for the
other event, making the inferred luminosities for the two transient
episodes less compatible with each other.

Having ruled out most of the usual suspects for extragalactic
transient events, we are left with two models: the \spock system
is either a RN or an LBV.  There are observed examples from both of
these classes that can separately match the three primary
observational characteristics: the peak luminosity (\Lpk), light curve
decline time (\t2), and recurrence timescale (\trec).  However, we
have found that when we bring all three of those constraints together
the RN model must be stretched to extreme physical limits. The
inferred peak luminosity of the \spock events would require a primary
white dwarf that is extraordinarily close to the Chandresekhar limit.
The uniquely rapid recurrence timescale would imply a mass transfer
rate from the secondary star that is remarkably fast ($>10{-7} \Msun$
yr${-1}$).  It is unclear whether these extremes are even physically
compatible as a RN system, as such a rapid mass transfer onto such a
massive white dwarf would likely result in stable nuclear burning at
the surface, and would therefore not lead to explosive burning
episodes that could be observed as rapid transient events.

Our preferred explanation for the \spock events is that we have
observed two distinct eruptive episodes from a massive LBV star.  The
light curve shape is consistent with rapid LBV eruptions seen in
systems such as SN 2009ip and NGC 3432-LBV1.  The peak luminosity and
recurrence timescale are also within the bounds of what has been
observed from nearby LBVs.  The \spock episodes may have been among
the fastest and most luminous of any rapid LBV events yet
observed. However, no other rapid LBV outbursts have yet been observed
with such a high cadence, so the detailed light curve shape can not be
rigorously compared against other events.  In this scenario, the
\spock LBV system would most likely have exhibited multiple eruptions
over the last few years, but most of them were missed, as they landed
within the large gaps of the \HST Frontier Fields imaging program.

We speculate that the very luminous and very fast \spock transients
may be driven by extreme mass eruption events or an extreme form of
stellar pulsation.  Both of these mechanisms are likely to occur in
LBV progenitor stars, but we do not have a robust model for precisely
how LBV eruptions are generated. This is a topic in need of
significant theoretical work, with the end goal being a comprehensive
physical model that accommodates both the \etacar-like great eruptions
and the S Dor-type variation of LBVs.  The \spock events are extreme
in several dimensions, and will only add to this theoretical
challenge.

The theoretical objective could be aided by additional observational
constraints on the \spock system.  If, as we have argued, this source
is indeed a recurrent system, then an observing campaign over $\sim$4
months should reveal at least one more transient event in one of the
three host galaxy images.  Capturing another transient episode with
similarly high cadence imaging could definitively resolve our
remaining classification ambiguities.  If the \spock source is a RN,
then we would expect the recurrence timescale, the light curve shape,
and the peak luminosity to all be relatively consistent from episode
to episode (as is the case for the rapid recurrence RN M31N 2008a-12).
In contrast, for an LBV system we would expect to see much more
variety in all of those observational characteristics.

If a sustained monitoring campaign on \macs0416 reveals no new
transient events over the span of a year or more, then we would need
to revise our classification.  Although it is quite possible for an
LBV star to undergo a long quiescent period, an absence of new
eruptions could also be an indication that the \spock events were
indeed the result of a terminal stellar explosion, such as a kilonova
or a .Ia supernova. In that case, improved lens models would be
necessary, to more rigorously examine the implications of concluding
that the 8-month separation between events was due entirely to the
lensing time delay.

An observing campaign that does catch additional transient episodes
would have another significant windfall.  If multiple events in
separate images of the host galaxy can be definitively matched, then
we could extract a measure of the gravitational lensing time delay.
Given the very sharp structure of the light curve, we anticipate that
these time delay measurements could be extremely precise, perhaps
constraining the delay to within a few hours.  This would offer a
powerful test of the lens models for this cluster, and in particular
could be a very sensitive check for systematic biases.  When a
suitable observational program is enacted, we would urge lens modelers
to make a concerted effort to optimize their time delay estimates in
advance of any time delay measurements, in order to provide blind
predictions for the most robust empirical test.

