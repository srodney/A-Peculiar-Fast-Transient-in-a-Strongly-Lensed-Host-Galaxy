\section{Conclusions}
\label{sec:Conclusions}

The peculiar transients \spockone and \spocktwo behind \macs0416 have
presented an intriguing puzzle.  It is immediately apparent that these
rapidly evolving transients do not fit neatly within any of the common
categories of explosive or eruptive astrophysical sources. To classify
this unusual pair of events and interrogate their physical basis, we
have combined the constraints from five models of the lensing cluster,
high-resolution imaging and spectroscopy of the host galaxy, and
high-cadence \HST sampling of the two light curves.  Although this
observational evidence is limited by significant uncertainties, we
have derived three key observable parameters for the \spock system:
(1) both events reached a peak luminosity that was within an order of
magnitude of 10$^{41}$ erg s$^{-1}$ ($M_V=-14$); (2) the rise and
decline timescales for both events was $\sim$2--5 days in the rest
frame; (3) the two events are most likely spatially coincident but not
coincident in time---with a separation of 3-5 months in the rest frame.

The most significant uncertainty in the luminosity (item 1) comes from
the gravitational lens modeling.  A lensing critical curve of the
\macs0416 cluster passes within 3\arcsec of both \spock events, which
makes it very difficult to pin down a prediction for the lensing
magnification.  From our 5 independent lens models we derive a
plausible range of magnifications ranging from $\mu\sim$10 to 100. In
spite of this large uncertainty, our luminosity constraint is still
sufficient to rule out stellar flares on the low end, and most types
of supernovae on the high end.

A second major uncertainty affecting both our constraints on the peak
luminosity and on the rise and decline times of the light curves comes
from the absence of photometric sampling immediately around the peak
of each light curve.  The rapidly evolving transient episodes were not
completely captured by the high-cadence imaging sequence of the HFF
program, leaving gaps of 3 and 14 days for \spockone and \spocktwo,
respectively.  Nevertheless, using simplistic assumptions about the
light curve behavior, we can reliably constrain the time each event
took to decline by 2 magnitudes to $\t2<6$ days---significantly
faster than almost all categories of explosive stellar phenomena.

%The lens models used here are diverse, but deliver a consistent
%story for the plausible range of time delays 
The diverse lens models analyzed here all agree that the gravitational
lensing time delay between the two positions of the \spock events is
far shorter than the observed time difference of $\sim$8 months.  This
leads to the conclusion that the two events are from a single {\em
  recurrent} source.  This conclusion could be assailed by moving the
location of the lensing critical curve, so that it is significantly
closer to one or the other of the two events.  This would drive up the
difference in the lensing potential between the two \spock positions,
and could lengthen the lensing time delay, which is directly
proportional to the lensing potential difference.  However, this would
also substantially increase the inferred lensing magnification of one
event, and simultaneously decrease $\mu$ for the other event, making
the inferred luminosities for two transient episodes incompatible.



\todo{quickly summarize the exclusion of the kN and .Ia models based
  on recurrence}
From these constraints 
We have examined these two models in comparison to three primary
observational constraints: the light curve decline time (\t2), the
peak luminosity (\Lpk), and the recurrence timescale (\trec).

We have found that the RN model is physically untenable. The light
curve shape is consistent with RN systems in our own galaxy and M31,
but the peak luminosity and recurrence timescale are at (or beyond)
the limits of the known RN population.  Indeed, these latter two
observables are, at best, only barely compatible with the physical
limits of the RN model. The inferred peak luminosity of the \spock
events would require a primary white dwarf that is extraordinarily
close to the Chandresekhar limit.  The uniquely rapid recurrence
timescale would imply a mass transfer rate from the secondary star
that is remarkably fast ($>10{-7} \Msun$ yr${-1}$).  It is unclear
whether these two extremes are even physically compatible as a RN
system, as such a rapid mass transfer onto such a massive white dwarf
would likely result in stable nuclear burning at the surface, and
would therefore not lead to explosive burning episodes that could be
observed as rapid transient events.

Our preferred explanation for the \spock events is that we have
observed two distinct eruptive episodes from a massive LBV star. 
\todo{Reiterate the phenomenoligical consistency}

We have found that the \spock events are compatible with the observed
properties of known LBV eruptions from the local universe.
\todo{Reiterate the observational and physical consistency}
The very luminous and very fast \spock
transients may be driven by extreme mass eruption events, an extreme
form of stellar pulsation, or may be caused by a different mechanism
entirely.

These events will only add to the theoretical challenge of deriving a
physical model that accommodates both the great eruptions and the S
Dor-type variation of LBVs.

\todo{Close with a prescription for resolving the classification with
  regular monitoring of the field, and getting measurable time delays
  that could provide a rigorous test of the cluster mass models}

