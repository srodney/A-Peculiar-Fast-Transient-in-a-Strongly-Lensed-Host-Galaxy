\section{Discussion}\label{sec:Discussion}

We have found that three models offer plausible explanations for the
\spock events: (1) these transients are due to a RN that reaches an
extraordinarily high peak luminosity, (2) they are separate rapid
outbursts of an LBV star, or (3) they are each the result of the
rapidly changing magnification of a massive star crossing over one or
more lensing caustics.  Our preferred explanation for the \spock
events is that we have observed two distinct eruptive episodes from a
massive LBV star.  The light curve shape is consistent with rapid LBV
eruptions seen in systems such as SN 2009ip and NGC 3432-LBV1.  The
peak luminosity and recurrence timescale are also within the bounds of
what has been observed from nearby LBVs.  The \spock episodes may have
been among the fastest and most luminous of any rapid LBV events yet
observed. However, the rapid outbursts of LBV stars in the local
universe have never yet been observed with such a high cadence, so the
detailed light curve shape of the \spock events cannot be rigorously
compared against other events.  In this scenario, the \spock LBV
system would most likely have exhibited multiple eruptions over the
last few years, but most of them were missed, as they landed within
the large gaps of the \HST Frontier Fields imaging program.

We speculate that the very luminous and very fast \spock transients
may be driven by extreme mass eruption events or an extreme form of
stellar pulsation.  Both of these mechanisms are likely to occur in
LBV progenitor stars, but there is not yet a consensus model that
explains precisely how LBV eruptions are generated. This is a topic in
need of significant theoretical work, with the end goal being a
comprehensive physical model that accommodates both the \etacar-like
great eruptions and the S Dor-type variation of LBVs.  The \spock
events are extreme in several dimensions, and should add a useful
benchmark for this theoretical challenge.

