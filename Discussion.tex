\section{Discussion}\label{sec:Discussion}

We have examined three plausible explanations for the \spock events:
(1) they were separate rapid outbursts of an LBV star, (2) they were
surface explosions from a single RN, or (3) they were each caused by
the rapidly changing magnification as two unrelated massive stars
crossed over lensing caustics. We can not make a definitive choice
between these hypotheses, principally due to the scarcity of
observational data and the uncertainty in the location of the
lensing critical curves.

If there is just a single critical curve for a source at $z=1$ passing
between the two \spock locations then our preferred explanation for
the \spock events is that we have observed two distinct eruptive
episodes from a massive LBV star.
%The light curve shape is consistent
%with rapid LBV eruptions seen in systems such as SN 2009ip and NGC
%3432-LBV1.  The peak luminosity and recurrence timescale are also
%within the bounds of what has been observed from nearby LBVs.  In this
%case, the \spock episodes may have been among the fastest and most
%luminous of any rapid LBV events yet observed. However, the rapid
%outbursts of LBV stars in the local universe have never yet been
%observed with such a high cadence, so the detailed light curve shape
%of the \spock events cannot be rigorously compared against other
%events.
In this scenario, the \spock LBV system would most likely
have exhibited multiple eruptions over the last few years, but most of
them were missed, as they landed within the large gaps of the \HST
Frontier Fields imaging program.
%We speculate that the very luminous and very fast \spock transients
%may be driven by extreme mass eruption events or an extreme form of
%stellar pulsation.  Both of these mechanisms are likely to occur in
%LBV progenitor stars, but there is not yet a consensus model that
%explains precisely how LBV eruptions are generated. This is a topic in
%need of significant theoretical work, with the end goal being a
%comprehensive physical model that accommodates both the \etacar-like
%great eruptions and the S Dor-type variation of LBVs.
The \spock events would be extreme LBV outbursts in several
dimensions, and should add a useful benchmark for the outstanding
theoretical challenge of developing a comprehensive physical model
that accommodates both the \etacar-like great eruptions and the S
Dor-type variation of LBVs.

If instead the \macs0416 lens has multiple critical curves that
intersect both \spock locations, then the third proposal of a
microlensing-generated transient would be preferred.  Stellar caustic
crossings have not been observed before, but the analysis of a likely
candidate behind the MACSJ1149 cluster suggests that massive cluster
lenses may generate such events more frequently than previously
expected \citep{Kelly:2017}. To resolve the uncertainty of the \spock
classification will require refinement of the lens models to more
fully address systematic biases and more tightly constrain the path of
the critical curve.  High-cadence monitoring of the \macs0416 field
would also be valuable, as it could catch future LBV eruptions or
microlensing transients at or near these locations.
