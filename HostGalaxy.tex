\subsection{Host Galaxy}\label{sec:HostGalaxy}

To examine whether the two transients originated from the same
physical location in the source plane, we looked for differences in
the properties of the \spock host galaxy at the location of each
event.  We first used the technique of ``pixel-by-pixel'' SED fitting
as described in \citet{Hemmati:2014} to determine rest-frame colors
and stellar properties in a single resolution element of the \HST
imaging data.  For this purpose we used the deepest possible stacks of
\HST images, comprising all available data except those images where
the transient events were present.  The resulting maps of stellar
population properties are shown in Figure~\ref{fig:HostProperties}.
Table~\ref{tab:HostProperties} reports measurements of the three
derived stellar population properties (color, mass, age) from host
images 11.1, 11.2 and 11.3.  In 11.1 and 11.2 these measurements were
extracted from the central pixel at the location of each of the two
\spock events.  Assuming the lensing magnification here ranges from
$\mu=10$ to 100 (see Section~\ref{sec:LensingModels}), this corresponds
to a size on the source plane between 6 and 600 pc$^2$.  For host
image 11.3 we report the stellar population properties derived from
the pixel at the center of the galaxy, because the lens models do not
have sufficient precision to map the \spock locations to specific
positions in image 11.3.  With a magnification of $\sim$3 to 5, this
extraction region covers roughly 2000 to 6000 pc$^2$.

\begin{deluxetable}{lccc}
  \tablewidth{\linewidth}
  \tablecolumns{6}
  \tablecaption{Properties of the local stellar population in the \spock host galaxy, from SED fitting.}
  \tablehead{ {Host image:} & \colhead{11.1} & \colhead{11.2} & \colhead{11.3}\\
{Location:}   & \colhead{\spocktwo} & \colhead{\spockone} & \colhead{center}}
\startdata
$(U-V)_{\rm rest}$            & 0.69$^{+0.2}_{-0.05}$  & 0.52$^{+0.15}_{-0.10}$      & 0.39$\pm$0.05  \\
$\log[\Sigma (M_*/\Msun)]$  & 7.14 $\pm$ 0.15   & 7.14 $\pm$ 0.15     & 7.04 $\pm$ 0.10   \\
Age (Gyr)                   & 0.292$\pm$0.5 &   0.290$\pm$0.5 &  0.292$\pm$0.5  
\enddata
\label{tab:HostProperties}
\end{deluxetable}

The reported uncertainties for these derived stellar properties in
Table~\ref{tab:HostProperties} reflect only the measurement errors
from the SED fitting, and do not attempt to quantify potential
systematic biases.  Such biases could arise, for example, from color
differences in the background light, which is dominated by the cluster
galaxies and varies significantly across the \MACS0416 field.  Such a
bias might shift the absolute values of the parameter scales for any
given host image (e.g., making the galaxy as a whole appear bluer,
more massive and younger). However, the gradients across any single
host image are unlikely to be driven primarily by such systematics.

Figure~\ref{fig:HostProperties} and Table~\ref{tab:HostProperties}
show that the measured values of the color, stellar mass, and age at
the two \spock locations are mutually consistent. Thus, it is
plausible to assume that the two positions map back to the same
physical location at the source plane.  Comparing those two locations
to the center of the galaxy as defined in image 11.3, we see only a
mild tension in the rest-frame U-V color. This comparison therefore
cannot quantitatively rule out the possibility that the two transient
events are located at the center of the galaxy. However, the maps
shown in Figure~\ref{fig:HostProperties} do show a gradient in both
U-V color and stellar age. For both images 11.1 and 11.2 the bluest
and youngest stars (U-V$\sim$0.3, $\tau\sim$280 Myr) are localized in
knots near the extreme ends of each image, well separated from either
of the \spock transient events.  In the less distorted host image 11.3
the bluer and younger stars are concentrated near the center. Taken
together, these color and age gradients suggest that the two
transients are not coincident with the center of their host galaxy.

In addition to the \HST imaging data, we also have spatially resolved
spectroscopy from the MUSE integral field data. The only significant
spectral line feature for the \spock host is the \ionline{O}{[ii]}
($\lambda\lambda$ 3727, 3729) doublet, observed at 7474 and 7478
\AA. Figure~\ref{fig:MUSEOIISequence} shows the observed
\ionline{O}{[ii]} lines at 10 positions along the length of the arc,
which comprises images 11.1 and 11.2.  At each position the lines were
extracted using apertures with a radius of 0\farcs6, so adjacent
extractions are not independent, but extractions at the two \spock
locations have no overlap. Each extraction has been normalized to show
a peak line flux at unity, so that the line profiles and the doublet
line ratios may be more easily compared.

Properties derived from these line fits are reported in
Table~\ref{tab:MuseLineFits}. The \ionline{O}{[ii]} lines do not exhibit
any discernible gradient across the host galaxy images in terms of the
wavelength of line centers, full width at half maximum, or the
intensity ratio of the two components of the doublet.  Thus, the
\ionline{O}{[ii]} measurements from MUSE cannot be used to
distinguish either \spock location from the other, or to definitively
answer whether either position is coincident with the center of the
host galaxy.  We conclude that it is plausible but not certain that
the two \spock events arose from the same physical location in the
host galaxy.

\begin{deluxetable*}{rllc ccc ccc c}
  \tablewidth{\linewidth}
  \tablecolumns{12}
  \tablecaption{Measurements of the \ionline{O}{[ii]} $\lambda\lambda$3626,3629 lines from \spock host galaxy images 11.1 and 11.2\tablenotemark{a}}
  \tablehead{ \colhead{Aperture} & \colhead{R.A.} & \colhead{Dec.} & \colhead{distance to} &  \multicolumn{3}{c}{\ionline{O}{[ii]} $\lambda$3726} & \multicolumn{3}{c}{\ionline{O}{[ii]} $\lambda$3729} & \colhead{Line}\\
    \colhead{ID} & \colhead{J2000} & \colhead{J2000} & \colhead{\spocktwo} & \colhead{Flux} & \colhead{$\lambda_{\rm center}$} & \colhead{FWHM} &
    \colhead{Flux} & \colhead{$\lambda_{\rm center}$} & \colhead{FWHM} & \colhead{Ratio}\\
    & \colhead{(degrees)} & \colhead{(degrees)} & \colhead{(Arcsec)} & \colhead{(erg\,s$^{-1}$\,cm$^{-2}$)} & \colhead{(\AA)} & \colhead{(\AA)} &
    \colhead{(erg\,s$^{-1}$\,cm$^{-2}$)} & \colhead{(\AA)} & \colhead{(\AA)}}
\startdata
    1  & 64.039371 &  -24.070450 &  -1.54 & 2.19e-18 &   7472.37 &  4.00 &    3.57e-18 &   7478.17 & 4.00 &      1.63\\      
    2  & 64.039218 &  -24.070345 &  -0.88 & 4.73e-18 &   7472.16 &  4.00 &    5.30e-18 &   7478.12 & 3.40 &      1.12\\
    3  & 64.039078 &  -24.070264 &  -0.30 & 5.05e-18 &   7472.29 &  4.00 &    6.10e-18 &   7478.27 & 3.73 &      1.21\\
    4  & 64.038921 &  -24.070163 &   0.39 & 4.22e-18 &   7472.19 &  4.00 &    5.74e-18 &   7478.08 & 3.59 &      1.36\\
    5  & 64.038785 &  -24.070078 &   0.97 & 3.86e-18 &   7472.25 &  4.00 &    6.56e-18 &   7478.19 & 4.00 &      1.70\\
    6  & 64.038637 &  -24.069958 &   1.65 & 4.80e-18 &   7472.51 &  4.00 &    5.42e-18 &   7478.07 & 2.69 &      1.13\\
    7  & 64.038501 &  -24.069865 &   2.24 & 4.60e-18 &   7472.57 &  3.43 &    5.74e-18 &   7478.17 & 3.20 &      1.25\\
    8  & 64.038352 &  -24.069752 &   2.92 & 4.70e-18 &   7472.54 &  3.54 &    6.22e-18 &   7478.16 & 2.95 &      1.32\\
    9  & 64.038229 &  -24.069648 &   3.50 & 3.26e-18 &   7472.83 &  2.80 &    5.79e-18 &   7478.16 & 2.84 &      1.77\\
   10  & 64.038076 &  -24.069532 &   4.19 & 2.44e-18 &   7473.01 &  2.57 &    3.22e-18 &   7478.10 & 2.73 &      1.32\\
Spo-1  & 64.038565 &  -24.069939 &   1.90 & 4.30e-18 &   7472.55 &  3.13 &    5.49e-18 &   7478.01 & 2.89 &      1.28\\
Spo-2  & 64.038998 &  -24.070241 &   0.00 & 4.37e-18 &   7472.46 &  4.00 &    6.10e-18 &   7478.22 & 3.79 &      1.40\\
\enddata

\tablenote{Properties of the of the \ionline{O}{[ii]} lines were
  derived from 1-D spectra extracted from the MUSE data cube at 10
  locations spaced 0\farcs6 apart along the length of the arc that
  comprises images 11.1 and 11.2 of the \spock host galaxy. Each
  extraction used an aperture of 0\farcs6 radius, centered on the
  mid-line of the host galaxy arc. The integrated line flux, observed
  wavelength of line center ($\lambda_{\rm center}$), and full width
  at half maximum (FWHM) were found by fitting a Gaussian profile to
  each component of the doublet.}
\label{tab:MuseLineFits}
\end{deluxetable*}


%#   ID   RA        DEC       Delta_spock2  OII]3726                                                                   OII]3729                                                                 OII]3729/OII]3726
%#                              Arcsec      Flux(erg/s/cm^2)  Center(Angstrom) Amplitude(erg/s/cm^2) FWHM(Angstrom)    Flux(erg/s/cm^2)  Center(Angstrom) Amplitude(erg/s/cm^2) FWHM(Angstrom)   
%#  ID    RA         DEC        d_spock2  flux3726           wave3726           amplitude3726       fwhm3726         flux3729           wave37269           amplitude3729       fwhm3729        fluxratio3729to3726



