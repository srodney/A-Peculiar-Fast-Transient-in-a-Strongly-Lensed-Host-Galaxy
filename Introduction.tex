\section{Introduction}\label{sec:Introduction}

In January and August of 2014, two unusual transient events were
observed in a strongly lensed galaxy at z=1.0054$\pm$0.0002.
Discovered by the FrontierSN team in Hubble Space Telescope (\HST)
observations from the Hubble Frontier Fields (HFF) program, these
events are designated \spockone and \spocktwo, and collectively
nicknamed ``Spock''.  Both transient episodes were faster and fainter
than any of the broad class of supernova-like transients.  They both
rose to a peak absolute optical/ultraviolet luminosity of $M\sim-14$
mag ($10^{41}$ erg s$^{-1}$) in only $\lesssim$5 rest-frame days, and
then faded away below detectability in roughly the same amount of
time.  The \spock events appeared in two adjacent arcs of a strongly
lensed galaxy that is multiply-imaged into at least 3 distinct images
by the gravitational potential of the galaxy cluster \MACS0416
(z=0.396).  Using five independent lens models of this cluster, we
find it is entirely plausible that the two events are {\it spatially}
coincident on the source plane, but very unlikely that they were also
{\it temporally} coincident.  Comparing the observational constraints
against existing categories of astrophysical transients, we find that
the most plausible classification is either as a recurrent nova (RN)
or a luminous blue variable (LBV).  The RN model would be strained to
its physical limits to accommodate the \spock observations. It would
require that the primary star is a white dwarf very close to the
Chandrasekhar mass limit, and that it is drawing mass from its
companion at an extremely efficient rate ($>10^{-7}$ \Msun yr$^{-1}$).
The LBV model is the most compatible, as it allows for a short
recurrence period, relatively high luminosity, and rapid light curve
rise and decline timescales. This model would imply that the \spock
system will most likely exhibit more eruptions in the near future.  A
high-cadence imaging campaign could catch these future eruptions,
allowing a clear test of this classification and providing an
opportunity for a very precise measurement of the gravitational
lensing time delay.


\TODO{discuss the HFF survey, the FrontierSN program, and the value of
  discovering strongly lensed transient events.}


In Sections 2-5 we develop three key observables for the \spock
events: (1) they are both more luminous than a classical nova, but
less luminous than a SN, reaching a peak luminosity of roughly
$10^{41}$ erg s$^{−1}$ ($M_V=−14$); (2) both transients exhibited fast
light curves, with rise and decline timescales of $\sim2--5$ days in
the rest frame; (3) it is likely that both events arose from the same
physical location but were not coincident in time---they were probably
separated by 3-5 months in the rest frame.
 

  
