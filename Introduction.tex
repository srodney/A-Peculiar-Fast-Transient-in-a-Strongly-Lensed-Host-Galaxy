\section{Introduction}\label{sec:Introduction}

The transient events designated \spockone and \spocktwo (collectively
nicknamed ``Spock'') appeared in Hubble Space Telescope (\HST) imaging
collected in January and August of 2014, respectively.  These images
were centered on the galaxy cluster \MACS0416\ (hereafter, MACS0416)
and were collected as part of the Hubble Frontier Fields (HFF) survey
(HST-PID:13496, PI:Lotz), a multi-cycle program observing 6 massive
galaxy clusters and associated ``blank sky'' parallel fields with very
deep imaging.

The \HST imaging and modeling of the gravitational lens lead to three
key observables for the \spock events: (1) they are both more luminous
than a classical nova, but less luminous than almost all supernovae
(SNe), reaching a peak luminosity of roughly $10^{41}$ erg s$^{−1}$
($M_V=−14$); (2) both transients exhibited fast light curves, with
rise and decline timescales of $\sim2--5$ days in the rest frame; (3)
it is likely that both events arose from the same physical location
but were not coincident in time---they were probably separated by 3-5
months in the rest frame.

Until recently, most surveys searching for extragalactic optical
transients have been optimized for the discovery of SNe, and
particularly for Type Ia SNe, because of their value as cosmological
probes \citep[e.g.,][and references therein]{Weinberg:2013}.  These
surveys have favored a cadence of several days between return visits,
with relatively short exposures to maximize the area of sky covered
while remaining sensitive to their primary targets---relatively bright
Type Ia SNe.  Although recent surveys are beginning to discover more
and more categories of rapidly changing optical transients
\citep[e.g.][]{Kasliwal:2011,Drout:2014} most programs remain largely
insensitive to transients with peak brightness and timescales
comparable to the \spock events \citep{Berger:2013}.  Future
wide-field observatories such as the Large Synoptic Survey Telescope
\citep[LSST,][]{Tyson:2002} will be much more efficient at discovering
such transients, and can be expected to reveal many new categories of
astrophysical transients.   

The HFF survey was not designed with discovery of peculiar
extragalactic transients as a core objective, but it has
unintentionally opened an early window of discovery for such events.
Very faint sources at relatively high redshift in these fields are
made detectable by the substantial gravitational lensing magnification
of the foreground galaxy clusters.  Very rapidly evolving sources are
also more likely to be found, due to the necessity of a rapid cadence
for repeat imaging in the HFF program.  These unusual characteristics
for an \HST survey contributed to the early detection and
characterization of SN Refsdal, the first SN that is strongly lensed
into multiple resolved images \citep{Kelly:2015a}.  The HFF imaging
program has also enabled a precise measurement of the lensing
magnification for SN Tomas, a high-redshift Type Ia SN
\citep{Rodney:2015a}.  Both of those SNe constitute fairly normal
astrophysical phenomena, and the possibility of discovering such
transients was anticipated at the start of the HFF program.  \spock,
however, appears to be {\it sui generis}.  No single astrophysical
model is clearly sufficient to explain all of the available
observational data, and the best available models require either an
extreme stellar source or a very unusual gravitational lensing
configuration.
