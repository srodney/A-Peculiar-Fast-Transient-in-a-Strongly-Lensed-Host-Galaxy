\section{Introduction}\label{sec:Introduction}

The Spock transient events---separately designated \spockone and
\spocktwo---appeared in Hubble Space Telescope (\HST) imaging
collected in January and August of 2014, respectively (Figure
\ref{fig:SpockDetectionImages}).  These observations were centered on
the galaxy cluster \MACS0416\ (hereafter, MACS0416) and were collected
as part of the Hubble Frontier Fields (HFF) survey (HST-PID:13496,
PI:Lotz), a multi-cycle program for deep imaging of 6 massive galaxy
clusters and associated ``blank sky'' fields observed in parallel
(Methods \ref{sec:Discovery}).

Combining the \HST imaging and lens models of the MACS0416 gravitational
lens leads to three key observables for the \spock events: (1) they are
both more luminous than a classical nova, but less luminous than
almost all supernovae (SNe), reaching a peak luminosity of roughly
$10^{41}$ erg s$^{−1}$ ($M_V=−14$); (2) both transients exhibited fast
light curves, with rise and decline timescales of $\sim$2--5 days in
the rest frame; and (3) it is likely that both events arose from the
same physical location but were not coincident in time---they were
probably separated by 3-5 months in the rest frame. These peculiar
transients thus present an intriguing puzzle: they are broadly
consistent with the expected behavior of stellar explosions (they each
exhibit a single isolated rise and decline in brightness), but they
can not be trivially classified into any of the common categories of
explosive or eruptive astrophysical transients.  

The HFF survey was not designed with the discovery of peculiar
extragalactic transients as a core objective, but it has
unintentionally opened an effective window of discovery for such
events.  Very faint sources at relatively high redshift ($z\gtrsim1$)
in these fields are made detectable by the substantial gravitational
lensing magnification from the foreground galaxy clusters.  Very
rapidly evolving sources are also more likely to be found, due to the
necessity of a rapid cadence for repeat imaging in the HFF program.
These unusual characteristics for an \HST survey enabled a precise
measurement of the lensing magnification for SN Tomas, a high-redshift
thermonuclear (Type Ia) SN \citep{Rodney:2015a}.  The HFF imaging
program also contributed to the detection and characterization of SN
Refsdal, a high-redshift core-collapse SN and the first
strongly-lensed SN observed with multiple resolved images
\citep{Kelly:2015a}.  The long-term monitoring campaign of SN Refsdal
then led to the discovery of the peculiar lensed transient {\it
  Icarus}, which is posited to be a stellar caustic crossing event
\citep{Kelly:2016}.  Even among this collection of rare transients,
the \spock events appear to be {\it sui generis}.  No single
astrophysical model is clearly sufficient to explain all of the
available observational data, and the best available models require
either an extreme stellar source or a very unusual gravitational
lensing configuration.

% Until recently, most surveys searching for extragalactic optical
% transients have been optimized for the discovery of SNe, and
% particularly for Type Ia SNe, because of their value as cosmological
% probes \citep[e.g.,][and references therein]{Weinberg:2013}.  These
% surveys have favored a cadence of several days between return visits,
% with relatively short exposures to maximize the area of sky covered
% while remaining sensitive to their primary targets---relatively bright
% Type Ia SNe.  Although recent surveys are beginning to discover more
% and more categories of rapidly changing optical transients
% \citep[e.g.][]{Kasliwal:2011,Drout:2014} most programs remain largely
% insensitive to transients with peak brightness and timescales
% comparable to the \spock events \citep{Berger:2013}.  Future
% wide-field observatories such as the Large Synoptic Survey Telescope
% \citep[LSST,][]{Tyson:2002} will be much more efficient at discovering
% such transients, and can be expected to reveal many new categories of
% astrophysical transients.   
