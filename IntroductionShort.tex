Until recently, most surveys searching for extragalactic optical
transients have been optimized for the discovery of SNe, and
particularly for Type Ia SNe, because of their value as cosmological
probes\cite{Weinberg:2013}.  These surveys have favored a cadence of
several days between return visits, with relatively short exposures to
maximize the area of sky covered while remaining sensitive to their
primary targets---relatively bright Type Ia SNe.  Although recent
surveys are beginning to discover more and more categories of rapidly
changing optical transients\cite{Kasliwal:2011a,Drout:2014} most
programs remain largely insensitive to transients with peak brightness
and timescales comparable to the \spock events\cite{Berger:2013b}.
Future wide-field observatories such as the Large Synoptic Survey
Telescope\cite{Tyson:2002} will be much more efficient at discovering
such transients, and can be expected to reveal many new categories of
astrophysical transients.

The Spock transient events---separately designated \spockone and
\spocktwo---represent a largely unexpected addition to the discovery
space of extragalactic transients.  As shown in
Fig.~\ref{fig:SpockDetectionImages}, they appeared in Hubble Space
Telescope (\HST) imaging collected as part of the Hubble Frontier
Fields (HFF) survey (HST-PID:13496, PI:Lotz), a multi-cycle program
for deep imaging of 6 massive galaxy clusters and associated ``blank
sky'' fields observed in parallel.  \HST is not an efficient
wide-field survey telescope, and the HFF survey was not designed with
the discovery of peculiar extragalactic transients as a core
objective.  However, the HFF program has unintentionally opened an
effective window of discovery for such events.  Very faint sources at
relatively high redshift ($z\gtrsim1$) in these fields are made
detectable by the substantial gravitational lensing magnification from
the foreground galaxy clusters.  Very rapidly evolving sources are
also more likely to be found, due to the necessity of a rapid cadence
for repeat imaging in the HFF program.

