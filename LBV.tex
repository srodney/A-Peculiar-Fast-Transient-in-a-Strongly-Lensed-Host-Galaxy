\subsection{Luminous Blue Variable Light Curve Comparison}
\label{sec:LBVlightcurves}

Figure~\ref{fig:LBVLightCurveComparison} presents a direct comparison
of the observed \spock light curves against the light curves of the
two LBVs that have well-studied rapid eruptions: SN 2009ip and
NGC3432-LBV1. The brief outbursts of these LBVs have been less finely
sampled than the two \spock events, but the available data show a wide
variety of rise and decline times, even for a single object over a
relatively narrow time window of a few months.



\subsection{LBV Build-up timescale}\label{sec:LBVbuildup}

The ``build-up'' timescale \citep{Smith:2011b} matches the radiative
energy released in an LBV eruption event with the radiative energy
produced during the intervening quiescent phase:

\begin{equation}
  \label{eqn:trad}
t_{\rm rad} = \frac{E_{\rm rad}}{L_{\rm qui}} = \t2 \frac{\xi\Lpk}{L_{\rm qui}},
\end{equation}

\noindent where $L_{\rm qui}$ is the luminosity of the LBV progenitor
star during quiescence.

The \spock events are not resolved as individual stars in their
quiescent phase, so we have no useful constraint on the quiescent
luminosity. Thus, instead of using a measured quiescent luminosity to
estimate the build-up timescale, we assume that $t_{\rm rad}$ for
\spock corresponds to the observed rest-frame lag between the two
events, roughly 120 days (this accounts for both cosmic time dilation
and a gravitational lensing time delay of $\sim$40 days). Adopting
$\Lpk=10^{41}$ erg s$^{-1}$ and $\t2=2$ days (see
Figure~\ref{fig:PeakLuminosityDeclineTime}), we infer that the
quiescent luminosity of the \spock progenitor would be $L_{\rm
  qui}\sim10^{39.5}$ erg s${-1}$ ($M_V\sim-10$).


%Rapid transient episodes in LBVs may
%best be explained by a sudden ejection of an optically thick shell
%\citep[e.g.,][]{Smith:2010, Smith:2011b}, or by some form of S
%Dor-type variability \citep{Weis:2005, VanDyk:2006, Foley:2011}, which
%may be driven by stellar pulsation rather than mass ejection
%\citep{VanGenderen:1997, VanGenderen:2001}.  For massive stars such as
%\etacar at its great eruption and the rapidly varying SN 2009ip, the
%effective photospheric radius during eruption must have been
%comparable to the orbit of Saturn \citep[$10^{14}$
%  cm;][]{Davidson:1997, Smith:2011b, Foley:2011}.  With observed
%photospheric velocities of order 500 km s$^{-1}$ for such events, the
%dynamical timescale of the extended photosphere is on the order of
%tens to hundreds of days.  The very rapid light curves of both \spock
%events would push down the lower limit of this range to just a few
%days.  If these are indeed LBV eruptions, then they are near the
%extreme limit of what is physically possible for such massive stellar
%eruptions.




%To examine the temperature and total energy output, we first make a
%set of (admittedly unfounded) assumptions: (1) the two outbursts had a
%very similar SED; (2) the last observed epoch for each event
%corresponds to the same phase relative to the true epoch of peak
%brightness; and (3) the lensing magnifications for the two events are
%the same.  These simplifying assumptions allow us to jointly apply the
%optical observations of \spockone and the NIR observations of
%\spocktwo as constraints on the SED in any given epoch.  We then set
%an assumption for the epoch of peak brightness, make another
%assumption for the magnification of both events, and then fit a
%blackbody to the resulting extrapolated SED. From this blackbody fit
%we derive a temperature and integrate to get an estimate of the
%pseudo-bolometric luminosity.  The resulting inferred physical
%parameters are plotted in Figure~\ref{fig:DerivedPhysicalParameters}.

