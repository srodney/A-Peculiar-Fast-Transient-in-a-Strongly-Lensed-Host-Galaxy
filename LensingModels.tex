\subsection{Gravitational Lens Models.}\label{sec:LensingModels}


%That same transient episode would have appeared at
%different times in host galaxy images 11.1 and 11.3, due primarily to
%the \citet{Shapiro:1964} delay.  The $\Delta t_{\rm NW:SE}$ column in
%Table~\ref{tab:LensModelPredictions} gives the model predictions for
%the number of days between the appearance of the \spockone\ event in
%the NW host image (11.2) and the date when it should have been
%observable in the adjacent SE host image (11.1).  The opposite value
%would give the time difference between the August 2014
%\spocktwo\ event and its expected appearance in host image 11.2.
%Table~\ref{tab:LensModelPredictions} also reports the predicted time
%delay (in years) between appearance in the NW host image 11.2 and the
%more widely separated image 11.3.

The seven lens models used to provide estimates of the plausible range
of magnifications and time delays are as follows:

\begin{itemize}
\item{\it CATS:} The model of \citeref{Jauzac:2014}, version 4.1,
  generated with the {\tt LENSTOOL} software
  (\url{http://projects.lam.fr/repos/lenstool/wiki})\citep{Jullo:2007}
  using strong lensing constraints.  This model parameterizes cluster
  and galaxy components using pseudo-isothermal elliptical mass
  distribution (PIEMD) density profiles\citep{Kassiola:1993,
    Limousin:2007}.
\item{\it GLAFIC:} The model of \citeref{Kawamata:2016}, built using
  the {\tt GLAFIC} software
  (\url{http://www.slac.stanford.edu/~oguri/glafic/})\citep{Oguri:2010b}
  with strong-lensing constraints. This model assumes simply
  parametrized mass distributions, and model parameters are
  constrained using positions of more than 100 multiple images.
\item{\it GLEE:} A new model built using the {\tt GLEE}
  software\citep{Suyu:2010b, Suyu:2012} with the same strong-lensing
  constraints used in \citeref{Caminha:2017}, representing mass
  distributions with simply parameterized mass profiles.
\item{\it GRALE:} A free-form, adaptive grid model developed using
  the GRALE software tool\citep{Liesenborgs:2006, Liesenborgs:2007,
    Mohammed:2014, Sebesta:2016}, which implements a genetic algorithm
  to reconstruct the cluster mass distribution with hundreds to
  thousands of projected Plummer\citet{Plummer:1911} density profiles.
\item{\it SWUnited:} The model of \citeref{Hoag:2016}, built using the
  {\tt SWUnited} modeling method\citep{Bradac:2005, Bradac:2009}, in
  which an adaptive pixelated grid iteratively adapts the mass
  distribution to match both strong- and weak-lensing constraints.
  Time delay predictions are not available for this model.
\item{\it WSLAP+:} Created with the {\tt WSLAP+} software
  (\url{http://www.ifca.unican.es/users/jdiego/LensExplorer})\citep{Sendra:2014}:
  Weak and Strong Lensing Analysis Package plus member galaxies (Note:
  no weak-lensing constraints were used for this \MACS0416 model).
\item{\it ZLTM:} A model with strong- and weak-lensing constraints,
  built using the ``light-traces-mass'' (LTM)
  methodology\citep{Zitrin:2009a, Zitrin:2015}, first presented for
  \MACS0416 in \citeref{Zitrin:2013a}.
\end{itemize}

Early versions of the {\it SWUnited}, {\it CATS}, {\it ZLTM} and {\it
  GRALE} models were originally distributed as part of the Hubble
Frontier Fields lens modeling project
(\url{https://archive.stsci.edu/prepds/frontier/lensmodels/}), in
which models were generated based on data available before the start
of the HFF observations to enable rapid early investigations of lensed
sources. The versions of these models applied here are updated to
incorporate additional lensing constraints.  In all cases the lens
modelers made use of strong-lensing constraints (multiply imaged
systems and arcs) derived from \HST imaging collected as part of the
CLASH program\cite{Postman:2012}). These
models also made use of spectroscopic redshifts in the cluster
field\cite{Mann:2012, Christensen:2012, Grillo:2015, Caminha:2017}.
Input weak-lensing constraints were derived from data collected at the
Subaru Telescope by PI K. Umetsu (in prep) and archival imaging.
%\citet{Priewe:2016} provides a more complete
%description of the methodology of model and a comparison of the
%magnification predictions and uncertainties across the entire
%\macs0416 field.


