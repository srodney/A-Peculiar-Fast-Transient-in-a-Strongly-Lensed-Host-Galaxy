\subsection{Gravitational Lens Models}\label{sec:LensingModels}


%That same transient episode would have appeared at
%different times in host galaxy images 11.1 and 11.3, due primarily to
%the \citet{Shapiro:1964} delay.  The $\Delta t_{\rm NW:SE}$ column in
%Table~\ref{tab:LensModelPredictions} gives the model predictions for
%the number of days between the appearance of the \spockone\ event in
%the NW host image (11.2) and the date when it should have been
%observable in the adjacent SE host image (11.1).  The opposite value
%would give the time difference between the August 2014
%\spocktwo\ event and its expected appearance in host image 11.2.
%Table~\ref{tab:LensModelPredictions} also reports the predicted time
%delay (in years) between appearance in the NW host image 11.2 and the
%more widely separated image 11.3.

The six lens models used to provide estimates of the plausible range
of magnifications and time delays are:

 \textcolor{red}{NOTE: THESE
  DESCRIPTIONS ARE PROBABLY INCORRECT.  MODELERS, PLEASE FIX AS
  NEEDED.}


\bigskip
\begin{itemize}
\item{\it Diego:} Created with the {\tt WSLAP+} software
  \citep{Sendra:2014}: Weak and Strong Lensing Analysis Package plus
  member galaxies (Note: no weak-lensing constraints used for this
  MACS J0416 model). Interactive online model exploration available at
  \url{http://www.ifca.unican.es/users/jdiego/LensExplorer}.
\item{{\it Jauzac:} The model of \citet{Jauzac:2014}, generated with
  the {\tt LENSTOOL} software
  \citep{Jullo:2007},\footnote{\url{http://projects.lam.fr/repos/lenstool/wiki}}}
  using strong- and weak-lensing constraints.  This model makes a
  light-traces-mass assumption and parameterizes cluster components
  using Navarro-Frenk-White (NFW) density profiles
  \citep{Navarro:1997}.
\item{\it Kawamata:} The model of \citet{Kawamata:2015}, built using the
  {\tt GLAFIC}
  software\footnote{\url{http://www.slac.stanford.edu/~oguri/glafic/}}
  with strong-lensing constraints.
\item{{\it Williams:} A free-form, adaptive grid model developed using
  the GRALE software tool \citep{Liesenborgs:2006, Liesenborgs:2007,
    Mohammed:2014, Sebesta:2016}, which implements a genetic algorithm
  to reconstruct the cluster mass distribution with projected Plummer
  \citeyear{Plummer:1911} density profiles.}
\item{{\it Zitrin:} A model with strong- and weak-lensing constraints,
  built using the ``light-traces-mass'' (LTM) methodology
  \citep{Zitrin:2009a,Zitrin:2015}, first presented for MACS0416 in
  \citet{Zitrin:2013a}.}
\end{itemize}
\bigskip    

The {\it Zitrin} model was originally distributed as part of the
Hubble Frontier Fields lens modeling project,\footnote{For more
  details, see
  \url{https://archive.stsci.edu/prepds/frontier/lensmodels/}} in
which models were generated based on data available before the start
of the HFF observations to enable rapid early investigations of lensed
sources. The {\it Jauzac} and {\it Williams} models are updated
versions of models developed for that HFF modeling effort by the CATS
and GRALE teams, respectively.  In all cases the lens modelers made
use of strong-lensing constraints (multiply-imaged systems and arcs)
derived from HST imaging collected as part of the CLASH program
(PI:Postman, HST PID:12459, \citealt{Postman:2012}). These models also
made use of spectroscopic redshifts in the cluster field from
\citet{Mann:2012}, \citet{Christensen:2012}, and \citet{Grillo:2015a}.
Input weak-lensing constraints also made use of data collected at the
Subaru Telescope by PI K. Umetsu (in prep) and archival imaging.
\citet{Priewe:2016} provides a more complete description of the
methodology of each model and a comparison of the magnification
predictions and uncertainties across the entire \macs0416 field.

Figure~\ref{fig:LensModelContours} presents probability distributions
derived from these models for the three magnifications and two time
delay values of interest.  These distributions were derived by
combining the Monte Carlo chains from the Jauzac, Oguri, Williams and
Zitrin models, with weighting applied to account for the different
number of model iterations in each chain. Four of the five models
agree that host image 11.3 is the leading image, appearing some 2--6
years before the other two images.  The models do not agree on the
arrival sequence of images 11.1 and 11.2: some have the NW image 11.2
as a leading image, and others have it as a trailing image.  However,
the models do consistently predict that the separation in time between
those two images should be roughly in the range of 1 to 60 days.

\todo{add reference to crit. curves fig}

%The angular separation of $1\farcs8$ between the \spock events
%corresponds to a physical separation of many tens of parsecs in the
%source plane.  A star could not traverse that distance in the
%$\sim$120 rest-frame days that separate the two \spock events.  Thus,
%even with a critical curve smeared out by the effects of the ICL, it
%would be impossible for a single star crossing a single caustic in the
%source plane to be responsible for both transients.
