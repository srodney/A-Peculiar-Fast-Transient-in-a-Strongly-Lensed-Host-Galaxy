\subsection{Gravitational Lens Models}\label{sec:LensingModels}


%That same transient episode would have appeared at
%different times in host galaxy images 11.1 and 11.3, due primarily to
%the \citet{Shapiro:1964} delay.  The $\Delta t_{\rm NW:SE}$ column in
%Table~\ref{tab:LensModelPredictions} gives the model predictions for
%the number of days between the appearance of the \spockone\ event in
%the NW host image (11.2) and the date when it should have been
%observable in the adjacent SE host image (11.1).  The opposite value
%would give the time difference between the August 2014
%\spocktwo\ event and its expected appearance in host image 11.2.
%Table~\ref{tab:LensModelPredictions} also reports the predicted time
%delay (in years) between appearance in the NW host image 11.2 and the
%more widely separated image 11.3.

The seven lens models used to provide estimates of the plausible range
of magnifications and time delays are:

\bigskip
\begin{itemize}
\item{{\it CATS:} The model of \citet{Jauzac:2014}, generated with
  the {\tt LENSTOOL} software
  \citep{Jullo:2007},\footnote{\url{http://projects.lam.fr/repos/lenstool/wiki}}}
  using strong lensing constraints.  This model makes a
  light-traces-mass assumption and parameterizes cluster and galaxy components
  using pseudo-isothermal elliptical mass distribution (PIEMD) density profiles
  \citep{Eliasdottir:2007, Limousin:2007}.
\item{\it GLAFIC:} The model of \citet{Kawamata:2016}, built using the
  {\tt
    GLAFIC}\footnote{\url{http://www.slac.stanford.edu/~oguri/glafic/}}
  software \citep{Oguri:2010b} with strong-lensing constraints. This
  model assumes simply parametrized mass distributions, and model
  parameters are constrained using positions of more than 100 multiple
  images.
\item{\it GLEE:} A new model built using the {\tt GLEE} software
  \citep{Suyu:2010b, Suyu:2012} with the same strong-lensing
  constraints used in \citet{Caminha:2017}, representing mass
  distributions with simply parameterized NFW profiles. 
\item{{\it GRALE:} A free-form, adaptive grid model developed using
  the GRALE software tool \citep{Liesenborgs:2006, Liesenborgs:2007,
    Mohammed:2014, Sebesta:2016}, which implements a genetic algorithm
  to reconstruct the cluster mass distribution with projected Plummer
  \citeyear{Plummer:1911} density profiles.}
\item{\it SWUnited:} The model of \citet{Hoag:2016}, built using the
  {\tt SWUnited} modeling method \citep{Bradac:2005, Bradac:2009}, in
  which an adaptive pixelated grid iteratively adapts the mass
  distribution to match both strong- and weak-lensing constraints.
  Time delay predictions are not available for this model.
\item{\it WSLAP+:} Created with the {\tt WSLAP+} software
  \citep{Sendra:2014}: Weak and Strong Lensing Analysis Package plus
  member galaxies (Note: no weak-lensing constraints used for this
  \MACS0416 model). Interactive online model exploration available at
  \url{http://www.ifca.unican.es/users/jdiego/LensExplorer}.
\item{{\it ZLTM:} A model with strong- and weak-lensing constraints,
  built using the ``light-traces-mass'' (LTM) methodology
  \citep{Zitrin:2009a,Zitrin:2015}, first presented for \MACS0416 in
  \citet{Zitrin:2013a}.}
\end{itemize}
\bigskip    

Early versions of the {\it SWUnited}, {\it CATS}, {\it ZLTM} and {\it
  GRALE} models were originally distributed as part of the Hubble
Frontier Fields lens modeling project,\footnote{For more details, see
  \url{https://archive.stsci.edu/prepds/frontier/lensmodels/}} in
which models were generated based on data available before the start
of the HFF observations to enable rapid early investigations of lensed
sources. The versions of these models applied here are updated to
incorporate additional lensing constraints.  In all cases the lens
modelers made use of strong-lensing constraints (multiply-imaged
systems and arcs) derived from \HST imaging collected as part of the
CLASH program (PI:Postman, HST-PID:12459,
\citealt{Postman:2012}). These models also made use of spectroscopic
redshifts in the cluster field from \citet{Mann:2012},
\citet{Christensen:2012}, \citet{Grillo:2015} and
\citet{Caminha:2017}.  Input weak-lensing constraints were derived
from data collected at the Subaru Telescope by PI K. Umetsu (in prep)
and archival imaging.
%\citet{Priewe:2016} provides a more complete
%description of the methodology of model and a comparison of the
%magnification predictions and uncertainties across the entire
%\macs0416 field.


Figure~\ref{fig:LensModelContours} presents probability distributions
derived from these models for the three magnifications and two time
delay values of interest.  These distributions were derived by
combining the Monte Carlo chains from the CATS, GLAFIC, GLEE, GRALE
and ZLTM models, with weighting applied to account for the different
number of model iterations in each chain. All five of these models
agree that host image 11.3 is the leading image, appearing some 3--7
years before the other two images.  The models do not agree on the
arrival sequence of images 11.1 and 11.2: some have the NW image 11.2
as a leading image, and others have it as a trailing image.  However,
the models do consistently predict that the separation in time between
those two images should be roughly in the range of 1 to 60 days. As
shown in Figure~\ref{fig:SpockDelayPredictions}, \spockone and
\spocktwo are inconsistent with these predicted time delays if one
assumes that they are delayed images of a single event.  However, if
these were independent events, then a time delay on the order of tens
of days between image 11.1 and 11.2 could have resulted in
time-delayed events that were missed by the \HST imaging of this
field.

%The angular separation of $1\farcs8$ between the \spock events
%corresponds to a physical separation of many tens of parsecs in the
%source plane.  A star could not traverse that distance in the
%$\sim$120 rest-frame days that separate the two \spock events.  Thus,
%even with a critical curve smeared out by the effects of the ICL, it
%would be impossible for a single star crossing a single caustic in the
%source plane to be responsible for both transients.

Because of the proximity of the critical curves in all models, the
predicted time delays and magnification factors are significantly
different if calculated at the model-predicted positions instead of
the observed positions.  For example, in the GLEE model series (GLEE-A
and GLEE-B) when switching from the observed to model-predicted
positions the arrival order of the NW and SE images flips, the
expected time delay drops from tens of days to $<$1 day, and the
magnifications decrease by 40-60\%.  However, the expected
magnifications and time delays between the events still fall within
the broad ranges summarized in Table~\ref{tab:LensModelPredictions}
and shown in Figure~\ref{fig:LensModelContours}.  Regardless of
whether the model predictions are extracted at the observed or
predicted positions of the \spock events, none of the lens models can
accommodate the observed 220-day time difference as purely a
gravitational lensing time delay.



\subsection{Lens Model Variations}\label{sec:LensModelVariations}

We used variations of several lens models to investigate how the
lensing critical curves shift under a range of alternative assumptions
or input constraints.  These variations highlight the range of
systematic effects that might impact the model predictions for the
\spock magnifications, time delays and proximity to the critical
curves.  Figure~\ref{fig:SpockCriticalCurves} shows the critical
curves for a source at $z=1$ (the redshift of the \spock host galaxy)
predicted by our seven baseline models, plus the four variations
described below.

The baseline CATS model reported in
Table~\ref{tab:LensModelPredictions} corresponds to the CATSv4.1 model
published on the STScI Frontier Fields lens model
repository\footnote{\url{https://archive.stsci.edu/pub/hlsp/frontier/macs0416/models/cats/v4.1/}}.
That model uses 178 cluster member galaxies, including a galaxy $<$5
arcsec south of the \spock host galaxy, which creates a local critical
curve that intersects the \spocktwo location.  Our CATS-var model is
an earlier iteration of the model, published on the STScI repository
as
CATSv4\footnote{\url{https://archive.stsci.edu/pub/hlsp/frontier/macs0416/models/cats/v4/}},
and includes only 98 galaxies identified as cluster members.  In this
variation the nearby cluster member galaxy is not included, so the
\spocktwo event is not intersected by a critical curve. However, the
\spockone event is approximately coincident with a the primary
critical curve of the \macs0416 cluster.  When the critical curve is
close to either \spock location, the magnifications predicted by the
CATS model are driven up to $\mu>100$.  However, the time delays
remain small, on the order of tens of days, and incompatible with the
observed 220-day gap.

The WSLAP-var model evaluates whether the cluster redshift
significantly impacts the positioning of the critical curve. In this
merging cluster, the northern brightest cluster galaxy (BCG) has a
slighter higher redshift than the southern BCG. The mean redshift of
the cluster is not precisely determined, since it is likely to be
aligned somewhat along the line of sight.  For the WSLAP-var model we
shift the assumed cluster redshift $z=0.4$ from the default $z=0.396$
(used in all the baseline models).  The shift in the critical curve is
noticeable, but not substantial, insofar as this change does not drive
the critical curve to intersect either or both of the \spock
locations.

The GLEE-var model is a multi-plane lens model that incorporates 13
galaxies with spectroscopic redshifts that place them either in the
foreground or background of the \macs0416 cluster.
Figure~\ref{fig:LineOfSightLenses} marks these 13 galaxies and
highlights two of them that appear in the foreground of the \spock
host galaxy and are close to the lines of sight to the \spock
transients. Both the foreground $z=0.0557$ galaxy and the
reconstructed position of the $z=0.9397$ galaxy have a projected
separation of $<$4\arcsec from the \spocktwo transient position.
Including these galaxies in the GLEE lensing model has a minor impact
on the magnifications and time delays, and also results in a shift of
the position of the critical curve--as can be seen by comparing the
GLEE-A and GLEE-B models in Figure~\ref{fig:SpockCriticalCurves}.  For the
GLEE model, incorporating these line-of-sight effects does not
substantially change the predicted magnifications or time delays, and
the predicted time delays are still incompatible with the observed gap
of 220 days between events.

The GLAFIC-var model examines whether it is plausible for a critical
curve to intersect both \spock locations---contrary to the baseline
assumption of a single critical curve subtending the \spock host
galaxy roughly midway between the two positions.  This model includes
a customized constraint, requiring that the magnification factors at
the \spock positons are $>1000$.  To achieve this, we independently
adjusted the mass scaling for the two nearest cluster member galaxies,
which are located just northeast and south of the \spock host galaxy
arc.  The mass of the northeast member galaxy was increased by
$\sim$30\% and that of the southern one by $\sim$60\%.  As a simple
check of the predicted morphology of the host galaxy, we placed a
source with a simple \citet{Sersic:1963} profile on the source
plane. The lensed image of that artificial source is an unbroken
elongated arc, reproducing the host galaxy image morphology reasonably
well.

For this modification of the GLAFIC lens model to be justified in a
statistical sense, the revised model should still accurately reproduce
the observed strong-lensing constraints across the entire cluster.
The $\chi^2$ statistic for the baseline GLAFIC model is 240, with 196
degrees of freedom ($\chi^2_\nu=1.2$), and yields an Akaike
information criterion \citep[AIC][]{Akaike:1974} of 676.  For the
GLAFIC-var model that forces multiple critical curves to intersect the
\spock locations, we get $\chi^2$=331 for 192 degrees of freedom
($\chi^2_\nu=1.7$) and AIC=769.  This suggests that the multiple
critical curve GLAFIC-var model is strongly disfavored by the
{\it positional} strong-lensing constraints that are used for both models.
However, we note that neither model incorporates the temporal
constraints of the observed time delay. 
