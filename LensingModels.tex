\subsection{Gravitational Lens Models}\label{sec:LensingModels}


%That same transient episode would have appeared at
%different times in host galaxy images 11.1 and 11.3, due primarily to
%the \citet{Shapiro:1964} delay.  The $\Delta t_{\rm NW:SE}$ column in
%Table~\ref{tab:LensModelPredictions} gives the model predictions for
%the number of days between the appearance of the \spockone\ event in
%the NW host image (11.2) and the date when it should have been
%observable in the adjacent SE host image (11.1).  The opposite value
%would give the time difference between the August 2014
%\spocktwo\ event and its expected appearance in host image 11.2.
%Table~\ref{tab:LensModelPredictions} also reports the predicted time
%delay (in years) between appearance in the NW host image 11.2 and the
%more widely separated image 11.3.

The six lens models used to provide estimates of the plausible range
of magnifications and time delays are:

\bigskip
\begin{itemize}
\item{{\it CATS:} The model of \citet{Jauzac:2014}, generated with
  the {\tt LENSTOOL} software
  \citep{Jullo:2007},\footnote{\url{http://projects.lam.fr/repos/lenstool/wiki}}}
  using strong lensing constraints.  This model makes a
  light-traces-mass assumption and parameterizes cluster and galaxy components
  using psuedo-isothermal elliptical mass distribution (PIEMD) density profiles
  \citep{Eliasdottir:2007}.
\item{\it GLAFIC:} The model of \citet{Kawamata:2016}, built using
  the {\tt
    GLAFIC}\footnote{\url{http://www.slac.stanford.edu/~oguri/glafic/}}
  software \citep{Oguri:2010b} with strong-lensing constraints. This
  model assumes simply parametrized mass distributions, and model
  parameters are constrained using positions of more than 100 multiple
  images.
\item{{\it GRALE:} A free-form, adaptive grid model developed using
  the GRALE software tool \citep{Liesenborgs:2006, Liesenborgs:2007,
    Mohammed:2014, Sebesta:2016}, which implements a genetic algorithm
  to reconstruct the cluster mass distribution with projected Plummer
  \citeyear{Plummer:1911} density profiles.}
\item{\it SWUnited:} The model of \citet{Hoag:2016}, built using the
  {\tt SWUnited} modeling method \citep{Bradac:2005, Bradac:2009}, in
  which an adaptive pixelated grid iteratively adapts the mass
  distribution to match both strong- and weak-lensing constraints.
  Time delay predictions are not available for this model.
\item{\it WSLAP+:} Created with the {\tt WSLAP+} software
  \citep{Sendra:2014}: Weak and Strong Lensing Analysis Package plus
  member galaxies (Note: no weak-lensing constraints used for this
  MACS0416 model). Interactive online model exploration available at
  \url{http://www.ifca.unican.es/users/jdiego/LensExplorer}.
\item{{\it Z-LTM:} A model with strong- and weak-lensing constraints,
  built using the ``light-traces-mass'' (LTM) methodology
  \citep{Zitrin:2009a,Zitrin:2015}, first presented for MACS0416 in
  \citet{Zitrin:2013a}.}
\end{itemize}
\bigskip    

Early versions of the {\it SWUnited}, {\it CATS}, {\it Z-LTM} and {\it
  GRALE} models were originally distributed as part of the Hubble
Frontier Fields lens modeling project,\footnote{For more details, see
  \url{https://archive.stsci.edu/prepds/frontier/lensmodels/}} in
which models were generated based on data available before the start
of the HFF observations to enable rapid early investigations of lensed
sources. The versions of these models applied here are updated to
incorporate additional lensing constraints.  In all cases the lens
modelers made use of strong-lensing constraints (multiply-imaged
systems and arcs) derived from HST imaging collected as part of the
CLASH program (PI:Postman, HST PID:12459,
\citealt{Postman:2012}). These models also made use of spectroscopic
redshifts in the cluster field from \citet{Mann:2012},
\citet{Christensen:2012}, and \citet{Grillo:2015a}.  The {\it Diego}
and {\it Kawamata} models presented here made use of the catalog of
spectroscopic redshifts and multiply-imaged background galaxies from
\citet{Caminha:2016}.  Input weak-lensing constraints were derived
from data collected at the Subaru Telescope by PI K. Umetsu (in prep)
and archival imaging.
%\citet{Priewe:2016} provides a more complete
%description of the methodology of model and a comparison of the
%magnification predictions and uncertainties across the entire
%\macs0416 field.

The locations of critical curves for a source at $z=1$ (the
redshift of the \spock host galaxy) are shown in
Figure~\ref{fig:SpockCriticalCurves}. In some cases the best-fit model
predicts that a single critical curve passes between the two \spock
locations, as would be expected if the host arc is indeed composed of
two elongated images of the host galaxy. Other models can accommodate
a lensing scenario that brings multiple critical curves across the two
spock locations. 

Figure~\ref{fig:LensModelContours} presents probability distributions
derived from these models for the three magnifications and two time
delay values of interest.  These distributions were derived by
combining the Monte Carlo chains from the CATS, GLAFIC, GRALE and
Z-LTM models, with weighting applied to account for the different
number of model iterations in each chain. Four of the five models
agree that host image 11.3 is the leading image, appearing some 2--6
years before the other two images.  The models do not agree on the
arrival sequence of images 11.1 and 11.2: some have the NW image 11.2
as a leading image, and others have it as a trailing image.  However,
the models do consistently predict that the separation in time between
those two images should be roughly in the range of 1 to 60 days. As
shown in Figure~\ref{fig:SpockDelayPredictions}, \spockone and
\spocktwo are inconsistent with these predicted time delays if one
assumes that they are delayed images of a single event.  However, if
these were independent events, then a time delay on the order of tens of
days between image 11.1 and 11.2 could have resulted in time-delayed 
events that were missed by the HST imaging of this field.


%The angular separation of $1\farcs8$ between the \spock events
%corresponds to a physical separation of many tens of parsecs in the
%source plane.  A star could not traverse that distance in the
%$\sim$120 rest-frame days that separate the two \spock events.  Thus,
%even with a critical curve smeared out by the effects of the ICL, it
%would be impossible for a single star crossing a single caustic in the
%source plane to be responsible for both transients.
