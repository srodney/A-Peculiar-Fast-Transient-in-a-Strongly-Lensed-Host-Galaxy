\subsection{Light Curve Fitting}\label{sec:LightCurves}

Due to the rapid decline timescale, no observations were collected for
either event that unambiguously show the declining portion of the
light curve. Therefore, we must make some assumption for the shape of
the light curve in order to quantify the peak luminosity and the
corresponding timescales for the rise and the decline.  We first
approach this with a simplistic model that is piece-wise linear in
magnitude vs time.  Figure~\ref{fig:LinearLightCurveFits} shows
examples of the resulting fits for the two events.  For each fit we
use only the data collected within 3 days of the brightest observed
magnitude, which allows us to fit a linear rise separately for the
F606W and F814W light curves for \spockone and the F125W and F160W
light curves for \spocktwo. To quantify the covariance between the
true peak brightness, the rise time and the decline timescale, we use
the following procedure:

\begin{enumerate}
\item make an assumption for the date of peak, $t_{\rm pk}$;
\item measure the peak magnitude at $t_{\rm pk}$ from the linear fit
  to the rising light curve data;
\item assume the source reaches a minimum brightness (maximum
  magnitude) of 30 AB mag at the epoch of first observation after the
  peak;
\item draw a line for the declining light curve between the assumed
  peak and the assumed minimum brightness;
\item use that declining light curve line to measure the timescale for
  the event to drop by 2 magnitudes, $t_2$;
\item make a new assumption for $t_{\rm pk}$ and repeat.
\end{enumerate}

As shown in Figure~\ref{fig:LinearLightCurveFits}, the resulting
piece-wise linear fits are simplistic, but nevertheless approximately
capture the observed behavior for both events.  Furthermore, since
this toy model is not physically motivated, it allows us to remain
agnostic for the time being as to the astrophysical source(s) driving
these transients.  From these fits we can see that \spockone most
likely reached a peak magnitude between 25 and 26.5 AB mag in both
F814W and F435W, and had a decline timescale $t_2$ of less than 2 days
in the rest-frame. The observations of \spocktwo provide less
stringent constraints, but we see that it had a peak magnitude between
23 and 26.5 AB mag in F160W and exhibited a decline time of less than
seven days.  These fits also illustrate the generic fact that a higher
peak brightness corresponds to a longer rise time and a faster decline
timescale, independent of the specific model used.  Changes to the
arbitrary constraints we placed on these linear fits do not
substantially affect the results.  For example, the relationship
between peak brightness and decline time is not strongly affected by
adjusting the assumed maximum post-peak magnitude or changing the
number of pre-peak data points used for the rising light curve fits.

At any assumed value for the time of peak brightness this linear
interpolation gives an estimate of the peak magnitude. We then convert
that to a luminosity (e.g., $\nu L_\nu$ in erg s$^{-1}$) by first
correcting for the luminosity distance assuming a standard \LCDM
cosmology, and then accounting for an assumed lensing magnification,
$\mu$.  The range of plausible lensing magnifications ($10<\mu<100$)
is derived from the union of our six independent lens models (Methods,
Figure~\ref{fig:LensModelContours}).  This results in a grid of
possible peak luminosities for each event as a function of
magnification and time of peak.  As we are using linear light curve
fits, the assumed time of peak is equivalent to an assumption for the
decline time, which we quantify as $t_2$, the time over which the
transient declines by 2 magnitudes.



\subsection{Color Curves}\label{sec:ColorCurves}

At redshift $z=1$ the observed optical and infrared bands translate to
rest-frame ultraviolet (UV) and optical wavelengths, respectively.  To
derive rest-frame UV and optical colors from the observed photometry,
we start with the measured magnitude in a relatively blue band (F435W
and F606W for \spockone and F105W, F125W, F140W for \spocktwo).  We
then subtract the coeval magnitude for a matched red band (F814W for
\spockone, F125W or F160W for \spocktwo), derived from the linear fits
to those bands.  To adjust these to rest-frame filters, we apply K
corrections \citep[following][]{Hogg:2002}, which we compute by
defining a crude SED via linear interpolation between the observed
broad bands for each transient event at each epoch.  For consistency
with past published results, we include in each K correction a
transformation from AB to Vega-based magnitudes.  The resulting UV and
optical colors are plotted in Figure~\ref{fig:ColorCurves}.  Both
\spockone and \spocktwo show little or no color variation over the
period where color information is available.  
