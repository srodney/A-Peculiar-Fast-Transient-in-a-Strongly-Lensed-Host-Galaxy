\subsection{Intracluster Light}\label{sec:ICL}

To estimate the mass of intracluster stars along the line of sight to
the \spock events, we follow the procedure of Kelly et al. (in prep)
and Morishita et al. (in prep).  This entails fitting and removing the
surface brightness of individual galaxies in the field, then fitting a
smooth profile to the residual surface brightness of intracluster
light (ICL).  The surface brightness is then converted to a projected
stellar mass surface density by assuming a Chabrier
\citeyear{Chabrier:2003} initial mass function and an exponentially
declining star formation history.  For further details, see Kelly et
al. (in prep).  This procedure leads to an estimate for the
intracluster stellar mass of
$\log (\Sigma_{\star} / (M_{\odot}\,{\rm kpc}^{-2})) = 6.9\pm0.4$.
This is very similar to the value of $6.8^{+0.4}_{-0.3}$ inferred for
the probable caustic crossing star {\it Icarus} (Kelly et al., in
prep).


\subsection{Expected Timescale for Microlensing Events}\label{sec:Microlensing}

A commonly observed example of microlensing-induced transient effects
is when a bright background source (a quasar) by a galaxy-scale lens
\citep{Wambsganss:2001, Kochanek:2004}.  In this optically thick
microlensing regime, the lensing potential along the line of sight to
the quasar is composed of many stellar-mass objects.  Each compact
object along the line of sight generates a separate critical lensing
curve, resulting in a complex web of overlapping critical curves. As
all of these lensing stars are in motion relative to the background
source, the web of caustics will shift across the source position,
leading to a stochastic variability on timescales of months to years.
This scenario is inconsistent with the observed data, as the two
\spock events were far too short in duration and did not exhibit the
repeated ``flickering'' variation that would be expected from
optically thick microlensing.

For the cluster-scale lens relevant in the case of \spock, we should
expect to be in the optically thin microlensing regime.  This
situation is similar to the ``local'' microlensing light curves
observed when stars within our galaxy or neighboring dwarf galaxies
pass behind a massive compact halo object \citep{Paczynski:1986,
  Alcock:1993, Aubourg:1993, Udalski:1993}.  In this case, an isolated
microlensing event can occur if there is a background star (i.e., in
the \spock host galaxy) that is the dominant source of luminosity in
its environment. In practice this means that the source must be a very
bright O or B star with mass of order 10 \Msun.  Depending on its age,
the size of such a star would range from a few to a few dozen times
the size of the sun.  The net relative transverse velocity would be on
the order of a few 100 km s$^{-1}$, which is comparable to the orbital
velocity of stars within a galaxy or galaxies within a cluster.

In the case of a smooth cluster potential, the timescale $\tau$ for
the light curve of such a caustic crossing event is dictated by the
radius of the source, $R$, and the net transverse velocity, $v$, of
the source across the caustic
\citep{Chang:1979,Chang:1984,MiraldaEscude:1991}:

\begin{equation}
  \tau = 6\frac{R}{5\,\Rsun}\frac{300 {\rm km~ s}^{-1}}{v}~\rm{hr}
\label{eqn:caustic_crossing_time}
\end{equation}

\noindent Thus, for reasonable assumptions about the star's radius and
velocity, the timescale $\tau$ is on the order of hours to days, which is well
matched to the observed rise and decline timescales of the \spock
events.


