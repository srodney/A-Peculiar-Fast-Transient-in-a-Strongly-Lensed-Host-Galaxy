\subsection{Discovery}\label{sec:Discovery}

The transient \spock\ was discovered in \HST imaging collected as part
of the Hubble Frontier Fields (HFF) survey (HST-PID:13496, PI:Lotz), a
multi-cycle program observing 6 massive galaxy clusters and associated
``blank sky'' parallel fields.  Several \HST observing programs have
provided additional observations supplementing the core HFF program.
One of these is the FrontierSN program (HST-PID:13386, PI:Rodney),
which aims to identify and study explosive transients found in the HFF
and related programs.  The FrontierSN team discovered \spock\ in two
separate HFF observing campaigns on the galaxy cluster
\MACS0416.  The first was an imaging campaign
in January, 2014 during which the MACS0416 cluster field was observed
in optical bands using the Advanced Camera for Surveys Wide Field
Camera (ACS-WFC).  The second concluded in August, 2014, and imaged
the cluster with the infrared detector of \HST's Wide Field Camera 3
(WFC3-IR).

To discover transient sources, the FrontierSN team processes each new
epoch of \HST data through a difference imaging
pipeline,\footnote{\url{https://github.com/srodney/sndrizpipe}} using
archival \HST images to provide reference images (templates) which are
subtracted from the astrometrically registered HFF images. In the case
of MACS0416, the templates were constructed from images collected as
part of the Cluster Lensing And Supernova survey with Hubble (CLASH,
HST-PID:12459, PI:Postman). The resulting difference images are
visually inspected for new point sources, and any new transients of
interest (primarily supernovae, SNe) are followed up with additional
\HST imaging or ground-based spectroscopic observations as needed.  For
a more complete description of the operations of the FrontierSN
program, see \citet{Rodney:2015a}.


\subsection{Photometry}\label{sec:Photometry}

The follow-up observations for \spock\ included \HST imaging
observations in infrared and optical bands using the WFC3-IR and
ACS-WFC detectors, respectively. Tables~\ref{tab:spockonephot} and
\ref{tab:spocktwophot} present photometry of the \spock\ events from
all available \HST observations. The flux was measured on difference
images, first using aperture photometry with a 0\farcs3 radius, and
also by fitting with an empirical point spread function (PSF).  The
PSF model was defined using \HST observations of the G2V standard star
P330E, observed in a separate calibration program.  A separate PSF
model was defined for each filter, but owing to the long-term
stability of the \HST PSF we used the same model in all epochs.  All
of the aperture and PSF fitting photometry was carried out using the
{\tt PythonPhot} software package
\citep{Jones:2015}.\footnote{\url{https://github.com/djones1040/PythonPhot}}


\subsection{Host Galaxy Spectroscopy}\label{sec:Spectroscopy}

Spectroscopy of the \spock\ host galaxy was collected using three
instruments on the Very Large Telescope (VLT).  Observations with the
VLT's X-shooter cross-dispersed echelle spectrograph
\citep{Vernet:2011} were taken on October 19th, 21st and 23rd, 2014
(Program 093.A-0667(A), PI: J. Hjorth) with the slit centered on the
position of \spocktwo.  The total integration time was 4.0 hours for
the NIR arm of X-shooter, 3.6 hours for the VIS arm, and 3.9 hours for
the UVB arm.  The spectrum did not provide any detection of the
transient source itself (as we will see below, it had already faded
back to its quiescent state by that time).  However, it did provide an
unambiguous redshift for the host galaxy of $z=1.0054\pm0.0002$ from
\Ha\ and the \forbidden{O}{ii} doublet in data from the NIR and VIS
arms, respectively.  These line identifications are consistent with
two measures of the photometric redshift of the host: $z=1.00\pm0.02$
from the BPZ algorithm \citep{Benitez:2000}, and $z=0.92\pm0.05$ from
the EAZY program \citep{Brammer:2008}.  Both were derived from \HST
photometry of the host images 11.1 and 11.2, spanning 4350--16000 \AA.

Additional VLT observations were collected using the Visible
Multi-object Spectrograph \citep[VIMOS][]{LeFevre:2003}, as part of
the CLASH-VLT large program \citep[Program 186.A-0.798; P.I.:
  P. Rosati;][]{Rosati:2014}), which collected $\sim$4000 reliable
redshifts over 600 arcmin$^2$ in the \macs0416 field
\citep{Grillo:2015,Balestra:2016}.  These massively multi-object
observations could potentially have provided confirmation of the
redshift of the \spock host galaxy with separate spectral line
identifications in each of the three host galaxy images.  For the
\macs0416 field the CLASH-VLT program collected 1 hour of useful
exposure time in good seeing conditions with the Low Resolution Blue
grism.  Unfortunately, the wavelength range of this grism (3600-6700
\AA) does not include any strong emission lines for a source at
z=1.0054, and the signal-to-noise (S/N) was not sufficient to provide
any clear line identifications for the three images of the \spock host
galaxy.

The VLT Multi Unit Spectroscopic Explorer
\citep[MUSE;][]{Henault:2003,Bacon:2012} observed the NE portion of
the MACS0416 field---where the \spock host images are located---in
December, 2014 for 2 hours of integration time (ESO program
094.A-0115, PI: J.\,Richard).  These observations also confirmed the
redshift of the host galaxy with clear detection of the
\forbidden{O}{ii} doublet.  Importantly, since MUSE is an integral
field spectrograph, these observations also provided a confirmation of
the redshift of the third image of the host galaxy, 11.3, with a
matching \forbidden{O}{ii} line at the same wavelength
(\citealt{Caminha:2017}, Richard et al. in prep).

A final source of spectroscopic information relevant to \spock is the
Grism Lens Amplified Survey from Space \citep[GLASS; PID:
  HST-GO-13459; PI:T. Treu;][]{Schmidt:2014,Treu:2015a}. The GLASS
program collected slitless spectroscopy on the \macs0416 field using
the WFC3-IR G102 and G141 grisms on \HST, deriving redshifts for
galaxies down to a magnitude limit $H<23$.  As with the VLT VIMOS
data, the three sources identified as images of the \spock host galaxy
are too faint in the GLASS data to provide any useful line
identifications.  There are also no other sources in the GLASS
redshift catalog\footnote{\url{http://glass.astro.ucla.edu/}} that
have a spectroscopic redshift consistent with z=1.0054.

