\section{Other Astrophysical Models}

The RN explanation for the two \spock events is consistent with the
available data, but requires the source to be an extreme version of a
RN system.  It therefore behooves us to examine whether there are any
more mundane astrophysical models that can accommodate all of the
available observations.

\subsection{Stellar Explosions}

The most common category of transient sources found in deep
extragalactic surveys are SNe, and our searches of the HFF survey
fields have found dozens of SNe, including several with significant
lensing magnification \citep{Rodney:2015,Kelly:2015a}. All SNe and
SN-like transients are much brighter than RNe, so the relatively high
peak luminosity of \spock could be more easily accommodated by
invoking a SN model. However, the very rapid rise and fall of both
light curves (observation \ref{obs:timescale}) is incompatible with
any of the normal SN classes.  For both thermonuclear white dwarf
explosions (Type Ia) and the core-collapse explosions of massive stars
(Type Ib, Ic, and II) the optical light curve after reaching peak
brightness is primarily powered by the decay of radioactive
\NiFiftySix to \CoFiftySix, which leads to a minimum decline rate of
$\sim$0.1 mag day$^{-1}$.  For Type II SNe this decay time can be
extended into a plateau phase by the recession of the photosphere via
a recombination wave propagint inward through the ionized H of the
expanding outer shell.  In no case can a normal SN powered by the
\NiFiftySix decay chain exhibit the decline rate of $t_3<8$ days that
has been observed for \spock.  Furthermore, the observed peak
luminosities for both \spock events are too low for most Type I or
Type II SN, which peak at $\sim10^{42}$ to $10^{43}$ erg s$^{-1}$.

Stellar flares are another very common class of transient phenomena in
which the photosphere brightens very rapidly due to magnetic activity
in the stellar atmosphere.  We can dismiss this possibility because
the observed luminosity of both \spock events (observation
\ref{obs:luminosity}) is far too high for even the most energetic
stellar flares. The brightest flare stars release a total energy in
the range of $10^{33}$ to $10^{38}$ erg over a span of minutes to
hours \citep{Balona:2012,Karoff:2016}.

The peak luminosity is marginally compatible with several categories
of ``SN-like'' transients, such as luminous red novae, Calcium-rich
transients, kilonovae, .Ia explosions, and fast optical transients.
However, there are no examples of known transients from these
categories with a light curve that rises and declines as quickly as
the two \spock events.  Furthermore, none of these transient
categories have been observed to exhibit recurrent outbursts within a
single year. Thus, to match the two \spock events one would need to
invoke either two separate sources in the same galaxy or explain the
time separation wholly as a result of gravitational lensing.





\subsection{Single Explosion, Time Delayed}




\subsection{Two Separate Sources}

Invoking two separate and unrelated explosions (Scenario
\ref{case:twoexplosions}) would imply that the two events are {\it
  not} coincident at the source plane in either space or time.  This
is consistent with the lens models, which makes this scenario
initially more attractive.  However, removing the spatial coincidence
is problematic, because we must then assert that two very similar
transients independently appeared by chance in the same host galaxy in
the same year.  This assertion would be plausible if the explosions
were from a very common source, but the rapid light curve already
rules out all of the common categories of supernova explosions.  Since
we have not observed any similar fast transients in any other galaxy
throughout the Frontier Fields survey, to accept this scenario we
would have to conclude that the \spock host galaxy is a very unusual
physical environment that provided uniquely fertile ground for such
transients.

%No other \HST survey has reported a transient source with
%a light curve similar to the \spock events, meaning that these -- a
%scenario which strains credulity regardless of

%All five lens models indicate that the locations of
%the two events are within \TODO{How many arcsec?  how many pc?}.

% \TODO{Make a figure that plots the SED of this host galaxy at the two
%   spock locations and at the center of the host galaxy image 11.3.
%   Quantitatively assess the probability that the three SEDs are
%   consistent with being at the same location on the source plane. }


%\TODO{Evaluate the star formation and metallicity of this host galaxy
%  to determine whether the host is very unique.}

Nevertheless, let us accept the premise that this host has generated
two rare transients in the same year, and consider what rare
explosion categories could be separately consistent with the observed
light curves.

\TODO{Evaluate kilonova, .Ia and Fast Optical transient light curves
  compared to the Spock light curves}

\TODO{Evaluate the likelihood of two rare explosions appearing in the
  same lensed host galaxy, comparing to rates from PS1, etc.}


\subsection{Two Events from the Same Source}

The final alternative is to allow two separate explosive events
powered by the same astrophysical source (Scenario
\ref{case:recurrent}).  In this case the observed events must be
spatially coincident but not coincident in time, which is fully
consistent with all lens model constraints.  Any transient event that
appears at the \spockone\ location must also appear at the
\spocktwo\ location, separated in time by a gravitational lensing time
delay.  Our lens models suggest that the time delay would be on the
order of 10--50 days, so this scenario supposes that those
``gravitational echoes'' were simply not observed, as they landed in
one of the long periods without \HST observations on this field.


\todo{Discuss LBV eruptions}



