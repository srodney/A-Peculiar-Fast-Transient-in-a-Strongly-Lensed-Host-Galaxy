%\section{
%
%The severe distortion of the
%host galaxy images makes it impossible to identify the location of the
%host center in images 11.1 and 11.2 from the galaxy morphology. Any
%spatial reconstruction at the source plane from the lens models would
%not be not precise enough for a useful test.  However, since
%gravitational lensing is achromatic, if the \spock positions are
%coincident with the host galaxy center, then the color of the galaxy
%at each \spock location in images 11.1 and 11.2 should be consistent
%with the color at the center of the less distorted image 11.3.

The AGN hypothesis for \spock is disfavored for three primary reasons:
First, AGN that exhibit short-duration transient events also typically
exhibit slower variation on much longer timescales, which is not
observed at either of the \spock locations. Second, the spectrum of
the \spock host galaxy shows none of the broad emission lines that are
often (though not always) observed in AGN.  Third, an AGN would
necessarily be located at the center of the host galaxy.  In
Section~\ref{sec:HostGalaxy} we saw that there are minor differences
in the host galaxy properties (i.e. rest-frame U-V color and mean
stellar age) from the \spockone and \spocktwo locations to the center
of the host galaxy at image 11.3 Although by no means definitive, this
suggests that the \spock events were not located at the physical
center of the host galaxy, and therefore are not related to an
AGN. \todo{Update with MUSE results}
