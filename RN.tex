\subsection{RN Light Curve Comparison}\label{sec:RNLightCurves}

%Nova outbursts can exhibit decline times from
%$\sim$1 day to many months, so the timescale of the \spock light
%curves can easily be accommodated by the nova scenario.
%However, the rise time of the \spock events is somewhat out of the ordinary for
%nova outbursts.  In particular, for recurrent nova eruptions that
%decline rapidly ($t_2<10$ days) they tend to also reach peak
%brightness very quickly, on timescales $<1$ day
%\citep{Schaefer:2010}. The 2014 eruption of the rapid-recurrence nova
%M31N 2008a-12 reached maximum brightness in a little under 1 day
%\citep{Darnley:2015}.  However, the rise time for nova eruptions is
%poorly constrained, as rapid-cadence imaging is rarely secured until
%after an initial detection near peak brightness.  Unlike the situation
%with a kilonova light curve, there is no a priori physical expectation
%for an especially rapid rise to peak in nova light curves.

There are ten known RNe in the Milky Way galaxy, and seven of
these exhibit outbursts that decline rapidly, fading by two magnitudes
in less than ten days \citep{Schaefer:2010}. 
% U Sco, V2487 Oph, V394 CrA, T CrB, RS Oph, V745  Sco, and V3890 Sgr.
Figure~\ref{fig:RecurrentNovaLightCurveComparison} compares the \spock
light curves to a composite light curve (the gray shaded region),
which encompasses the V band light curve templates
\citep{Schaefer:2010} for all seven of these galactic RN events.  The
Andromeda galaxy (M31) also hosts at least one RN with a rapidly
declining light curve.  The 2014 eruption of this well-studied nova,
M31N 2008a-12, is shown as a solid black line in
Figure~\ref{fig:RecurrentNovaLightCurveComparison}, fading by 2 mags
in less than 3 days.  This comparison demonstrates that the rapid
decline of both of the \spock transient events is fully consistent
with the eruptions of known RNe in the local universe.

%Among the most luminous classical novae known, a similarly rapid
%decline time is not unheard of.  For example, the bright nova
%M31N-2007-11d had $t_2 = 9.5$ days \citep{Shafter:2009}.  The
%extremely luminous nova SN 2010U had $t_2 = 3.5 \pm 0.3$
%\citep{Czekala:2013}.  The nova L91 required at least 4 days to rise
%to maximum \citep{Shafter:2009}, and then declined with $t_2 = 6 \pm
%1$ days \citep{DellaValle:1991, Williams:1994, Schwarz:2001}.


\subsection{RN Luminosity and Recurrence Period}\label{sec:RNLuminosityRecurrence}

To examine the recurrence period and peak brightness of the \spock
events relative to RNe, we rely on a pair of papers that evaluated an
extensive grid of nova models through multiple cycles of outburst and
quiescence \citep{Prialnik:1995,Yaron:2005}.
Figure~\ref{fig:RecurrentNovaRecurrenceComparison} plots first the RN
outburst amplitude (the apparent magnitude between outbursts minus the
apparent magnitude at peak) and then the peak luminosity against the
log of the recurrence period in years.
% The observations for \spock 
% are shown in comparison to observed RNe (crosses) and theoretical
% models (circles) from \citet{Yaron:2005}.
For the \spock events we can only measure a lower limit on the
outburst amplitude, since the presumed progenitor star is unresolved,
so no measurement is available at
quiescence. Figure~\ref{fig:RecurrentNovaRecurrenceComparison} shows
that a recurrence period as fast as one year is expected only for a RN
system in which the primary WD is both very close to the Chandrasekhar
mass limit (1.4 \Msun) and also has an extraordinarily rapid mass
transfer rate ($\sim10^{-6}$ \Msun yr$^{-1}$).  The models of
\citet{Yaron:2005} suggest that such systems should have a very low
peak amplitude (barely consistent with the lower limit for \spock) and
a low peak luminosity ($\sim$100 times less luminous than the \spock
events).

The closest analog for the \spock events from the population of known
RN systems is the nova M31N\,2008a-12.  \citet{Kato:2015} provided a
theoretical model that can account for the key observational
characteristics of this remarkable nova: the very rapid recurrence
timescale ($<$1 yr), fast optical light curve ($\t2\sim2$ days), and
short supersoft x-ray phase \citep[6-18 days after optical
  outburst][]{Henze:2015a}.  To match these observations,
\citeauthor{Kato:2015} invoke a 1.38 \Msun white dwarf primary,
drawing mass from a companion at a rate of $1.6\times10^{-7}$ \Msun
yr$^{-1}$.  This is largely consistent with the theoretical
expectations derived by \citet{Yaron:2005}, and reinforces the
conclusion that a combination of a high mass white dwarf and efficient
mass transfer are the key ingredients for rapid recurrence and short
light curves. The one feature that cannot be effectively explained
with this scenario is the peculiarly high luminosity of the \spock
events -- even after accounting for the very large uncertainties. 
