\subsection{Rates}\label{sec:RatesMethods}

To derive a rough estimate of the rate of \spock-like transients, we
first define the set of strongly lensed galaxies in which a similarly
faint and fast transient could have been detected in the HFF
imaging. The single-epoch detection limit of the HFF transient search
was $m_{\rm lim}=26.7$ AB mag, consistent with the SN searches carried
out in the CLASH and CANDELS programs\cite{Graur:2014,Rodney:2014}.
For a transient with peak brightness $M_{V}>-14$ mag to be detected,
the host galaxy must be amplified by strong lensing with a
magnification $\mu>20$ at $z\sim1$, growing to $\mu>100$ at $z\sim2$.
Using photometric redshifts and magnifications derived from the GLAFIC
lens models the six HFF clusters, we find $N_{\rm gal}=6$ galaxies
that satisfy this criterion, with redshifts $0.5<z<1.5$ (Extended Data
Fig.~\ref{fig:StronglyLensedGalaxies}).

We then define the {\it control time}, $t_{c}$, for the HFF survey,
which gives the span of time over which each cluster was observed with
a cadence sufficient for detection of such rapid transients.  We
define this as any period in which at least two \HST observations were
collected within every ten day span. This effectively includes the
entirety of the primary HFF campaigns on each cluster, but excludes
all of the ancillary data collection periods from supplemental \HST
imaging programs. The average control time for an HFF cluster is
$t_{c}$=0.22 years (80 days).  Treating each \spock event as a
separate detection, we can derive a rate estimate using $R = 2 /
(N_{\rm gal}\,t_c)$.  This yields $R=1.5$ events galaxy$^{-1}$
year$^{-1}$.   

Future examination of the rate of such transients should consider the
total stellar mass and the star formation rates of the galaxies
surveyed, or use a projection of the lensed background area onto the
source plane to derive a volumetric rate.  Such analyses would require
a more detailed exploration of the impact of lensing uncertainties on
derived properties of the lensed galaxies and the lensed volume, and
this is beyond the scope of current work.

