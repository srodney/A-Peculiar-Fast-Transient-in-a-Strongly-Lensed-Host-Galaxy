\section{Recurrent Nova Model}
\label{sec:RecurrentNovaModel}

From the data presented in
Sections~\ref{sec:Observations}--\ref{sec:LensModels}, we conclude
that the two \spock events were most likely generated by two separate
outbursts from a single stellar source, such as a Recurrent Nova (RN)
system. A nova explosion can occur in a binary star system in which
the more massive star is a white dwarf that accretes matter from its
companion, which may be a main sequence dwarf or evolved giant star
overfilling its Roche Lobe. The white dwarf builds up a dense layer of
H-rich material on its surface until the high pressure and temperature
triggers nuclear fusion, resulting in a surface explosion
that causes the white dwarf to brighten by several orders of
magnitude, but does not completely disrupt the star. In a RN system,
the mass transfer from the companion to the white dwarf restarts after
the explosion, so the cycle may begin again and repeat after a period
of months or years.  As we demonstrate below, this model can
accommodate all of the available evidence.  However, we will see that
the luminosity and light curve of the \spock events would imply that
this is physically a very extreme RN system.

\subsection{Decline Rate}

A first line of evidence supporting the nova hypothesis comes from the
\spock light curves. Some RN light curves are similar in shape to the
\spock episodes, exhibiting a sharp rise ($<10$ days in the
rest-frame) and a similarly rapid decline.
Figure~\ref{fig:RecurrentNovaLightCurveComparison} compares the \spock
light curves to template light curves from RNe within our galaxy and
in M31.  There are 11 known RNe in the Milky Way galaxy, and 5 of
these exhibit outbursts that decline rapidly, fading by 2 magnitudes
in less than 10 days: U Sco, V2487 Oph, V394 CrA, T CrB, and V745 Sco.
The gray shaded region in
Figure~\ref{fig:RecurrentNovaLightCurveComparison} encompasses the V
band light curve templates for all 5 of these events, from
\citet{Schaefer:2010}.  The Andromeda galaxy (M31) also hosts at least
one RN with a rapidly declining light curve.  The 2014 eruption of
this well-studied nova, M31N 2008a-12, is shown as a solid black line
in Figure~\ref{fig:RecurrentNovaLightCurveComparison}, fading by 2
mags in less than 3 days.  This comparison demonstrates that the
sudden disappearance of both of the \spock transient events is fully
consistent with the eruptions of known RNe in the local universe.

\subsection{Recurrence}

The second reason to consider the RN model is that it provides a
natural explanation for having two separate explosions that are
coincident in space but not in time.  If \spock is a RN, then the two
observed episodes can be attributed to two distinct nova eruptions,
and the gravitational lensing time delay does not need to match the
observed 8 month separation between the January and August 2014
appearances.  Alternatively, one might suppose that the two \spock
events are actually two images of the same physical episode that
appear to us separately only because of the lensing delay, as was the
case for the 5 images of SN Refsdal \citep{Kelly:2015a,Kelly:2016}.
% For SN Refsdal
% the lens models were collectively very accurate in predicting the time
% delays between the 4 images in the Einstein cross configuration
% \citep{Treu:2015b,Rodney:2016} and the return as a fifth image
% \citep{Kelly:2016}. The lens modeling for \spock uses much of the same
% methodology, so there is no a priori reason to be suspicious of the
% time delay predictions.

However, we have seen in Section~\ref{sec:LensingModels} that none of
the \macs0416 lens models predict an 8 month time delay between
appearances in image 11.1 and 11.2.  This is represented in
Figure~\ref{fig:SpockDelayPredictions}, where we have plotted the
light curves for the two transient events, along with shaded vertical
bars marking the time delay predictions of all models.
%The lens models are broadly consistent with each other, predicting
%that the lensing time delay between images 11.1 and 11.2 is on the
%order of $\pm$60 days, far short of the 238 day lag that was observed
%between \spockone\ and \spocktwo.
To accept the alternative single-explosion explanation
for \spock, we would have to assume that a large systematic bias is
similarly affecting all of the lens models.  While we cannot rule out
such a bias, the consistency of the lens modeling makes a recurrent
explosion model more tenable.

Although {\it qualitatively} consistent, the RN model is strained by a
quantitative assessment of the recurrence period. If \spock is indeed
a RN at $z=1$, then the recurrence timescale in the rest-frame is
$120\pm30$ days ($3-5$ months), where the uncertainty accounts for the
$1\sigma$ range of modeled gravitational lensing time delays.  This
would be a singularly rapid recurrence period, significantly faster
than all 11 RNe in our own galaxy, which have recurrence timescales
ranging from 15 years (RS Oph) to 80 years (T CrB). For the 5 galactic
RNe with a rapidly declining outburst light curve (U Sco, V2487 Oph,
V394 CrA, T CrB, and V745 Sco), the median recurrence timescale is 21
years.  The fastest measured recurrence timescale belongs to the
Andromeda galaxy nova M31N 2008a-12, which exhibits a new outburst
every year. Although this M31 record-holder demonstrates that very
rapid recurrence is possible, classifying \spock as a RN would still
require a very extreme mass transfer rate to accommodate the $<1$ year
recurrence.

\subsection{Luminosity}


extremely luminous novae \citep{Czekala:2013}
was an Fe II nova, inconsistent with the usual picture of He/N novae as the brightest with the most massive WDs.

Most classical novae exhibit a well-documented maximum-magnitude versus
rate-of-decline (MMRD) relationship \citep{McLaughlin:1945}, in that
more luminous novae tend to have more rapidly declining light curves.

Furthermore, the spectroscopic classification
of Novae is also correlated with their luminosity and light curve
decline time: those showing prominent He/N features are brighter and
fade faster than those with spectra dominated by Fe II lines.

fast and faint novae don't follow the MMRD \citep{Kasliwal:2011}


\citet{Shafter:2011} examined 91 novae in M31 (both classical Novae
and RNe) and showed that the   These observations are consistent with the classification of 

consistent with the
short separation between observations of \spock. 


Shafter et al 2011:

``more luminous novae generally fade the fastest and [...]  He/N novae
are typically faster and brighter than their Fe II counterparts. In
addition, we find a weak dependence of nova speed class on position in
M31, with the spatial distribution of the fastest novae being slightly
more extended than that of slower novae.''


Recurrent novae make up roughly 25\% of the nova population
(masquerading as CNe \citep{Pagnotta:2014}.
