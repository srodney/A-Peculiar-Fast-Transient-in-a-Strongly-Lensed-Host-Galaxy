\section{Results}\label{sec:Results}

Photometry collected from the HST images (Methods
\ref{sec:Photometry}), reveals that the the two \spock\ exhibited very
similar light curves, as shown in Figure \ref{fig:LightCurves}.  Both
events reached a peak luminosity of $\sim$0.08 \microjansky, with the
entire observable rise and fall occurring in $<20$ days.  The host
galaxy for both \spock events is at a redshift of $z=1.0054\pm0.0002$
(Methods \ref{sec:Spectroscopy}).  After dividing the observed
timescales by ($1+z$) to correct for cosmic time dilation, we see that
both events lasted $<$10 days in the rest frame.

To interpret the observed light curves and the timing of these two
events, we use six independently constructed cluster mass models to
determine the impact of gravitational lensing from the MACS0416
cluster (Methods \ref{sec:LensingModels}).  The lensing scenario that
has been consistently adopted for this cluster assumes that the arc in
which the \spock events appeared consists of two mirror images of the
host galaxy, labeled 11.1 and 11.2 in
Figure~\ref{fig:SpockDetectionImages} \citep{Zitrin:2013a,
  Jauzac:2014, Johnson:2014, Richard:2014, Diego:2015a, Grillo:2015a,
  Hoag:2016, Sebesta:2016}.  This implies that a single critical curve
passes roughly mid-way between the two observed \spock locations.

As reported in Table~\ref{tab:LensModelPredictions}, our six lens
models predict absolute magnification values between about $\mu=10$
and $\mu=100$ for both events.  This wide range is due primarily to
the close proximity of the \spock\ events to the lensing critical
curve (the region of theoretically infinite magnification) for sources
at $z=1$.  Note that the magnifications for \spockone\ and
\spocktwo\ are highly correlated.  A variation of a given lens model
that moves the critical curve closer to the position of
\spockone\ would drive the magnification of that event much higher
(toward $\mu_1\sim100$), but that would also have the effect of moving
the critical curve farther from \spocktwo\, which would necessarily
drive its magnification downward (toward $\mu_2\sim10$). From each
model we also extract two time delay predictions, given in
Table~\ref{tab:LensModelPredictions}.  We report all time delays
relative to the \spockone\ event, which appeared in January 2014 in
host image 11.2.

%\renewcommand{\arraystretch}{1.5}
\begin{deluxetable}{lccccc}\label{tab:LensModelPredictions}
\tablewidth{\linewidth} \tablecolumns{6} \tablecaption{Lens model
  predictions for time delays and
  magnifications. \label{tab:LensModelPredictions}} 
\tablehead{
  \colhead{Model} & \colhead{$\Delta t_{\rm NW:SE}$} & \colhead{$\Delta
    t_{\rm NW:11.3}$} & \colhead{$\mu_1$} & \colhead{$\mu_2$} &
  \colhead{$\mu_3$}\\ \colhead{} & \colhead{(days)} &
  \colhead{(years)} & \colhead{} & \colhead{} & \colhead{} }
\startdata
SWunited & \nodata & \nodata & 38 $\pm8$ & 12.8 $\pm0.8$ & 2.9 $\pm0.1$\\[0.5em]
WSLAP+ & -48$\pm$10 & 0.8 & 35$\pm$20 & 30$\pm$20 & \nodata\\[0.5em]
%Jauzac-A & 29$\pm3$ & -2.2$\pm0.1$ & 36 $^{+4}_{-3}$ & 17 $\pm1$ & 4.5 $\pm0.1$\\[0.5em]
CATS & 28$^{+5}_{-7}$ & -2.2$^{+0.1}_{-0.2}$ & 37 $\pm3$ & 18 $\pm2$ & 4.6 $\pm0.1$\\[0.5em]
%Jauzac-C & 2.5$^{+0.8}_{-1.0}$ & -3.0$\pm0.1$ & 37 $^{+4}_{-3}$ & 16 $^{+1}_{-2}$ & 3.2 $\pm0.05$\\[0.5em]
GLAFIC & 4.1$^{+5.5}_{-3.4}$ & -5.0$^{+0.5}_{-0.6}$ & 29 $^{+43}_{-10}$ & 84 $^{+103}_{-38}$ & 3.0 $^{+0.2}_{-0.2}$\\[0.5em]
GRALE & -10$^{+1}_{-7}$ & -2.5$^{+1.0}_{-3.1}$ & 13 $^{+11}_{-6}$ & 12 $^{+9}_{-5}$ & 3.1 $^{+2.2}_{-0.9}$\\[0.5em]
Z-LTM & 42$^{+13}_{-9}$ & -3.7$\pm0.3$ & 90 $^{+61}_{-27}$ & 32 $^{+8}_{-10}$ & 3.6 $^{+0.2}_{-0.5}$\\
%% New Jauzac model predictions
%$\mu_{\rm NW}$ = 0.2 $^{+0.1}_{-0.1}$
%$\mu_{\rm SE}$ = 0.4 $^{+0.4}_{-0.1}$
%$\mu_{\rm 11.3}$ = 1.1 $^{+0.4}_{-0.2}$
%$\Delta t_{\rm NW:SE}$~[days] = 3.3 $^{+1.1}_{-1.1}$ days
%$\Delta t_{\rm NW:11.3}$~[yrs] = -4.6 $^{+0.1}_{-0.1}$ years
\enddata
\tablecomments{
  Each lens model is identified by the name of the modeling method.
  %lead modeler or
  %the principal investigator of the modeling team.
  Time delays give the predicted delay relative to an
  appearance in the NW host image, 11.2. Positive (negative) values indicate the
  NW image is the leading (trailing) image of the pair.
}
\end{deluxetable}
%\renewcommand{\arraystretch}{1.}

\subsection{Ruling Out Common Astrophysical Transients}

There are several categories of
astrophysical transients that cannot accommodate the light curve
characteristics of the \spock transients.  We may first dismiss any of
the category of {\it periodic} sources (e.g. Cepheids, RR Lyrae, or
Mira variables) that exhibit regular changes in flux due to pulsations
of the stellar photosphere. These variable stars do not exhibit sharp,
isolated transient episodes that could match the \spock light curve
shapes. Stellar flares provide another very common source for optical
transient events, but the total energy released by even the most
extreme stellar flare falls far short of the observed energy release
from the \spock transients \citep{Balona:2012,Karoff:2016} .

We can also rule out active galactic nuclei (AGN), in which
brief transient episodes (a few days in duration) may be observed from
X-ray to infrared wavelengths \citep[e.g.][]{Gaskell:2003},
principally due to the quiescence of the \spock sources between the
two observed episodes and the absence of any of the broad emission
lines that are often (though not always) observed in AGN.  No x-ray
emitting point source was detected in 7 epochs of imaging from
the Swift and Chandra x-ray space telescopes, collected from 2009 to
2014 (Methods~\ref{sec:Xray}).  This includes the Chandra imaging on
August 31, 2014 (MJD=56900), which was coeval with the peak
of IR emission from \spocktwo, observed with HST.

Many types of stellar explosions can generate isolated
transient events, and a useful starting point for classification of
such objects is to examine their position in the phase
space of peak luminosity versus decline time \citep[see,
  e.g.,][]{Kulkarni:2007}. We have inferred a range of plausible
luminosities and decline times for the \spock events using linear
interpolation of the observed photometry (Methods
\ref{sec:LightCurves}), with corrections for the luminosity distance
(assuming a standard \LCDM cosmology), and accommodating the range of
viable lensing magnifications ($10<\mu<100$) derived from the cluster
lens models. This results in two-dimensional constraints on \Lpk and
the decline timescale \t2 (the time over which the transient declines
by 2 magnitudes). The \spockone and \spocktwo events are largely
consistent with each other, and if both events are representative of a
single system (or a homogeneous class) then the most likely peak
luminosity and decline time (the region with the most overlap) would
be $L_{\rm pk}\sim10^{41}$ ergs/s and $t_2\sim1.8$ days.

As shown in Figure~\ref{fig:PeakLuminosityDeclineTime}, the relatively
low peak luminosities and the very rapid rise and fall of both \spock
light curves are incompatible with all of the common categories of
stellar explosions. This includes the thermonuclear explosions of
white dwarf stars as Type Ia SNe, and the heterogeneous class of
core collapse SNe.  Less well-understood classes are also
ruled out, such as Superluminous SNe \citep{Gal-Yam:2012,Arcavi:2016},
Type Iax SNe \citep{Foley:2013a}, fast optical transients
\citep{Drout:2014}, Ca-rich SNe
\citep{Filippenko:2003,Perets:2011,Kasliwal:2012}, and Luminous Red
Novae \citep[also called intermediate luminosity red
  transients;][]{Munari:2002,Kulkarni:2007,Kasliwal:2011b}.

The SN-like transients that come closest to matching the observed
characteristics of the two \spock events are the ``kilonova'' class
(also called a ``macronova'' or ``mini-supernova'') and the ``.Ia''
class.  Kilonovae are a theorized category of optical transients that
may be generated by the merger of a neutron star (NS) binary. Such a
NS+NS merger can drive a relativistic jet that may be observed as a
Gamma Ray Burst (GRB) and would emit gravitational waves.  These may
also be accompanied by a very rapid optical light curve (the kilonova
component) that is driven by the radioactive decay of r-process
elements in the ejecta \citep{Li:1998,Kulkarni:2005}.  The .Ia class
is due to He shell explosions that are expected to arise from AM Canum
Venaticorum (AM CVn) binary star systems undergoing He mass transfer
onto a white dwarf primary star \citep{Warner:1995,
  Nelemans:2005,Bildsten:2007}.  The \spock light curves exhibited a
slower rise time than is expected for a kilonova event
\citep[e.g.,][]{Metzger:2010,Barnes:2013,Kasen:2015}, and a faster
decline time than is anticipated for a .Ia event
\citep[e.g.,][]{Shen:2010}.  Furthermore, the \spock observations are not
well matched by the few examples of observed kilonova candidates
\citep{Perley:2009,Tanvir:2013} or .Ia candidates
\citep{Kasliwal:2010, Perets:2010, Poznanski:2010}.  However, there is
enough uncertainty about the diversity of light curves generated by
these rare explosions that we cannot dismiss these models on the
basis of light curve characteristics alone.

In addition to the incompatibility of the light curve shape, another
problem for all of the catastrophic stellar explosions discussed so
far is that they cannot explain the appearance of {\it repeated}
transient events.  As shown in Table~\ref{tab:LensModelPredictions}, none of
the MACS0416 lens models predict an 8 month time delay between
appearances in host galaxy images 11.1 and 11.2, suggesting that
\spocktwo is not a time-delayed image of \spockone \citep[as was the
  case for the 5th image of SN Refsdal;][]{Kelly:2015a,Kelly:2016}.

Neither the kilonova nor the .Ia explosion models can accommodate
repeated events from the same source. The kilonova progenitor
systems are completely disrupted at explosion.  For .Ia events, even
if an AM CVn system could produce repeated He shell flashes of similar
luminosity, the period of recurrence would be of order $10^5$ yr,
making these effectively non-recurrent sources.  Thus, to reconcile
any cataclysmic explosion model with the two observed \spock events
we would need to invoke a highly serendipitous occurrence of two
unrelated peculiar explosions in the same host galaxy in the same
year. It is difficult to quantitatively assess the likelihood of such
an occurrence, as there are no measured rates of .Ia or kilonovae.  In
a study of very fast optical transients with the Pan-STARRS1 survey,
\citet{Berger:2013b} derived a limit of $\lesssim0.05$ Mpc$^{-3}$
yr$^{-1}$ for transients reaching $M\approx -14$ mag on a timescale of
$\sim$1 day.  This limit, though several orders of magnitude higher
than the constraints on novae or SNe, is sufficient to make it
exceedingly unlikely that two unrelated fast optical transients would
appear in the same galaxy in a single year.  Furthermore, we have
observed no other transient events with similar luminosities and light
curve shapes in high-cadence surveys of 5 other Frontier Fields
clusters. Indeed, all other transients detected in the primary HFF
survey have been fully consistent with normal SNe.  Thus, we have no
evidence to suggest that transients of this kind are common enough to
be observed twice in a single galaxy in a single year.

%As an alternative solution, one might assert that a systematic bias is
%similarly affecting all of the lens models, the two events are two
%images of the same explosion, appearing to us separately only because
%of a gravitational lensing time delay.  While we cannot rule out such
%a bias, the consistency of the lens modeling makes this scenario less
%tenable.

%\subsection{Non-explosive Astrophysical Transients}\label{sec:OtherTransients}

Although the two events were most likely not {\it temporally}
coincident, all six lens models indicate that it is entirely
plausible for the two \spock events to be {\it spatially} coincident: they
can be mapped back to the same physical location at the source
plane. This is supported by the fact that the host galaxy colors and
spectral indices at each \spock location are indistinguishable within
the uncertainties (Methods \ref{sec:HostGalaxy}).  
Thus, to accommodate all of the observations of the \spock events with
a single astrophysical source, we turn to two categories of
stellar explosion that are sporadically recurrent: luminous blue
variables (LBVs) and recurrent novae (RNe).

\subsection{Luminous Blue Variable}

The transient sources categorized as LBVs are the result of eruptions
or explosive episodes from massive stars ($>10$\Msun).  The class is
exemplified by prototypical examples such as P Cygni, $\eta$ Carinae
(\etacar), and S Doradus (S-Dor) \citep[for recent overviews of the
  LBV class, see][]{Smith:2011b, Kochanek:2012}.  Although most giant
LBV eruptions have been observed to last much longer than the \spock
events \citep{Smith:2011b}, some LBVs have exhibited repeated rapid
outbursts that are broadly consistent with the very fast \spock light
curves (Methods \ref{sec:LBVlightcurves}). Because of this commonly
seen stochastic variability, the LBV scenario does not have any
trouble accounting for the \spock events as two separate episodes.

Two well-studied LBVs that provide a plausible match to the observed
\spock events are ``SN 2009ip'' \citep{Maza:2009} and NGC3432-LBV1
\citep{Pastorello:2010}.  Both exhibited multiple brief transient
episodes over a span of months to years \citep[e.g.,][]{Miller:2009,
  Li:2009, Berger:2009, Drake:2010, Pastorello:2010}.  Unfortunately,
for these outbursts we have only upper limits on the decline
timescale, $t_2$, due to the relatively sparse photometric sampling.
Figure~\ref{fig:PeakLuminosityDeclineTime}b shows that both \spock
events are consistent with the observed luminosities and decline times
of these fast and bright LBV outbursts -- though the \spock events
would be among the most rapid and most luminous LBV eruptions ever
seen.

In addition to those relatively short and very bright giant eruptions,
most LBVs also commonly exhibit a slower underlying variability. P
Cygni and \etaCar, for example, slowly rose and fell in brightness by
$\sim$1 to 2 mag over a timespan of several years before and after
their historic giant eruptions.  Such variation has not been detected
at the \spock locations. Nevertheless, given the broad
range of light curve behaviors seen in LBV events, we cannot reject
this class as a possible explanation for the \spock system.

To explore some of the physical implications of an LBV
classification for the two \spock events, we first make a rough
estimate of the total radiated energy, which can be computed using the
decline timescale $t_2$ and the peak luminosity $L_{\rm pk}$ following
\citet{Smith:2011b}:

\begin{equation}
  \label{eqn:Erad}
  E_{\rm rad} = \zeta \t2 \Lpk,
\end{equation}

\noindent where $\zeta$ is a factor of order unity that depends on the
precise shape of the light curve.\footnote{Note that
  \citet{Smith:2011b} used $t_{1.5}$ instead of $t_2$, which amounts
  to a different light curve shape term, $\zeta$.}  Adopting
\Lpk$\sim10^{41}$ erg s$^{-1}$ and \t2$\sim$2 days (as shown in
Figure~\ref{fig:PeakLuminosityDeclineTime}), we find that the total
radiated energy is $E_{\rm rad}\sim10^{46}$ erg.  A realistic range
for this estimate would span $10^{44}<E_{\rm rad}<10^{47}$ erg, due to
uncertainties in the magnification, bolometric luminosity correction,
decline time, and light curve shape (in roughly that order of
importance). These uncertainties notwithstanding, our crude estimate
does fall well within the range of plausible values for the total
radiated energy of a major LBV outburst.

The precise physical mechanism for LBV outbursts is still not fully
understood \citep[e.g.][]{Smith:2006,Woosley:2007,Dessart:2010}, but
the canonical model is that LBV transient events are the optical
signature of an eruptive mass loss episode.  If this applies to the
\spock transients, then the energy budget must also include a
substantial amount of kinetic energy imparted to the ejected gas
shell. Without spectroscopic information from the \spock transients we
cannot place any realistic estimate on the kinetic
energy. Nevertheless, we can take the radiated energy as a rough lower
limit on the total energy release and consider what timescale would be
required for a massive star to build up that amount of energy (Methods \ref{sec:LBVbuildup}). This approach assumes that the energy
released in an LBV eruption is generated slowly in the stellar
interior and is in some way ``bottled up'' by the stellar envelope,
before being released in a rapid mass ejection.  By assuming that the
build-up timescale is comparable to the rest-frame time between the
two observed events, we estimate a quiescent luminosity of
$L_{\rm  qui}\sim10^{39.5} erg s^{-1}$ ($M_V\sim-10$).  This value is fully
consistent with the expected range for LBV progenitor stars (e.g.,
\etacar has $M_V\sim-12$ and the faintest known LBV progenitors such
as SN 2010dn have $M_V\sim-6$).


\subsection{Recurrent Nova}\label{sec:RNe}

Novae occur in binary systems in which a white dwarf star accretes
matter from a less massive companion, leading to a burst of nuclear
fusion in the accreted surface layer that causes the white dwarf to
brighten by several orders of magnitude, but does not completely
disrupt the star. In a recurrent nova (RN) system, the mass transfer
from the companion to the white dwarf restarts after the explosion, so
the cycle may begin again and repeat after a period of months or
years.

The light curves of many RN systems in the Milky Way are similar in
shape to the \spock episodes, exhibiting a sharp rise ($<10$ days in
the rest-frame) and a similarly rapid decline (see Methods section
\ref{sec:RNLightCurves}).  This is reflected in
Figure~\ref{fig:PeakLuminosityDeclineTime}, where novae are
represented by a grey band that traces the empirical constraints on
the maximum magnitude - rate of decline (MMRD) relation for classical
novae \citep{DellaValle:1995, Downes:2000, Shafter:2011,
  Kasliwal:2011a}.

The RN model can provide a natural explanation for having two separate
explosions that are coincident in space but not in time, as the two
observed \spock events can be attributed to two distinct eruptions
from the same RN system.  However, the recurrence timescale for \spock
in the rest-frame is $120\pm30$ days, which would be a singularly
rapid recurrence period for a RN system.  All RNe in our own galaxy
have recurrence timescales ranging from 15--80 years
\citep{Schaefer:2010}.  The fastest measured recurrence timescale
belongs to the Andromeda galaxy nova M31N 2008a-12, which has
exhibited a new outburst every year from 2009-2015
\citep{Tang:2014,Darnley:2014,Darnley:2015,Henze:2015,Henze:2015a}. Although
this M31 record-holder demonstrates that very rapid recurrence is
possible, classifying \spock as a RN would still require a very
extreme mass transfer rate to accommodate the $<1$ year recurrence.
The non-detection of the \spock events in x-ray imaging from the
Chandra space telescope (Methods~\ref{sec:Xray}) is also problematic
for the RN scenario, as the optical emission of a nova is commonly
followed by a supersoft x-ray phase that can last months to years
\citep[e.g.]{Hachisu:2006}.

Another major concern with the RN hypothesis is that the
two \spock events are substantially brighter than all known novae --
perhaps by as much as 2 orders of magnitude.  One might attempt to
reconcile the \spock luminosity more comfortably with the nova class
by assuming a significant lensing magnification for one of the two
events. This would drive down the intrinsic luminosity, perhaps to
$\sim10^{40}$ erg s$^{-1}$, on the edge of the nova region.  However,
this assumption implicitly moves the lensing critical curve to be
closer to the \spock event in question.  That pulls the critical curve
away from the other \spock position, which makes that second event
{\it more inconsistent} with observed nova peak luminosities.  

The combination of a rapid recurrence timescale and unusually high
peak luminosity for the \spock events is at odds with theoretical
expectations and empirical constraints for RNe (see Methods section
\ref{sec:RNLuminosityRecurrence}). Although the RN model is not
strictly ruled out, we can deduce that if the \spock transients are
caused by a single RN system, then that progenitor system would be the
most extreme white dwarf binary system yet known.

% As discussed above, modified lens models may allow for two critical
% curves crossing the host arc close to the \spock events
% (Methods~\ref{sec:Microlensing}).  This would make the RN hypothesis
% more plausible as the inferred lensing magnification would
% simultaneously increase for both \spockone and \spocktwo, thereby
% bringing the intrinsic luminosity of both events into line with the
% expectations for a rapidly recurrent nova. This would, however,
% require substantial fine-tuning of the lens models in order to
% accommodate multiple critical curves that still map the spock events
% back to the same location on the source plane.



\subsection{Microlensing}\label{sec:MicroLensing}

In the presence of strong gravitational lensing it is possible to
generate a transient event from lensing effects alone.  In this case
the background source has a steady luminosity but the relative motion
of the source, lens, and observer causes the magnification of that
source (and therefore the apparent brightness) to change rapidly with
time.  An isolated strong lensing event with a rapid timescale can be
generated when a background star crosses over a lensing critical
curve.  In the case of a star crossing the caustic of a smooth lensing
potential, the amplification of the source flux would increase
(decrease) with a characteristic $t^{-1/2}$ profile as it moves toward
(away from) the caustic. This slowly evolving light curve then
transitions to a very sharp decline (rise) when the star has moved to
the other side of the caustic \citep{Schneider:1986,
  MiraldaEscude:1991}.  With a more complex lens comprising many
compact objects, the light curve would exhibit a superposition of many
such sharp peaks \citep{Lewis:1993}.

The peculiar transient {\it Icarus}, observed behind the Hubble
Frontier Fields cluster MACS J1149.6+2223, has been proposed as the
first observed example of such a stellar caustic crossing event
(P. Kelly et al., in prep). Kelly et al. find that such events may be
expected to appear more frequently in strongly lensed galaxies that
have small angular separation from the center of a massive cluster. In
such a situation, our line of sight to the lensed background galaxy
passes through a dense web of overlapping micro-lenses caused by the
intracluster stars distributed around the center of the cluster. This
has the effect of ``blurring'' the magnification profile across the
cluster critical curve, making it more likely that a single (and rare)
massive star in the background galaxy gets magnified by the required
factor of $\sim10^5$ to become visible as a transient caustic crossing
event.  On this basis the \spock host galaxy images are suitably
positioned for caustic crossing transients, as they are seen through a
relatively high density of intracluster stars---comparable to that
observed for the {\it Icarus} transient (Methods \ref{sec:ICL}).

The characteristic timescale of a canonical caustic crossing event
would be on the order of hours or days (Methods
\ref{sec:Microlensing}), which is comparable to the timescales
observed for the \spock events. Gravitational lensing is achromatic as
long as the size of the source is consistent across the SED.  This
means that the color of a caustic crossing transient will be roughly
constant.  Using simplistic linear interpolations of the observed
light curves (Methods \ref{sec:LightCurves}) we find that the inferred
color curves for both \spock events are marginally consistent with
this expectation of an unchanging color (Methods
\ref{sec:ColorCurves}).

In the baseline lensing scenario adopted above---where a single
critical curve subtends the \spock host galaxy arc---these events
cannot plausibly be explained as stellar caustic crossings, because
neither transient is close enough to the single critical curve to
reach the required magnifications of $\mu\sim10^6$.  Some lens models
can, however, be modified so that instead of just two host images, the
lensed galaxy arc is made up of many more images of the host, with
multiple critical curves subtending the arc where the \spock events
appeared.  By tuning the assumed masses of cluster galaxies near the
\spock host, those multiple critical curves can be made to pass very
close to the positions of the two \spock transient events.  In the
{\it Kawamata} and {\it Diego} lens models, this alternative lensing
scenario requires that the masses of the two nearest cluster galaxies
are increased by $30-60\%$.  With this adjustment, both models can
reproduce the observed morphology of the HFF14Spo host galaxy as a
smooth, unbroken arc.  These model realizations imply magnifications
on the order of $\mu\sim1000$ for both \spock transients. If this
alternative lensing scenario is correct, then similar microlensing
transients would be expected to appear at different locations along
the host galaxy arc, instigated by new caustic crossing episodes from
different stars in the host galaxy.
