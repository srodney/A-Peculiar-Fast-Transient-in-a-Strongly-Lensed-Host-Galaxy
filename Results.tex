\section{Results}\label{sec:Results}

To evaluate the impact of gravitational lensing from the \MACS0416
cluster on the observed light curves and the timing of these two
events, we use seven independently constructed cluster mass models.
These models indicate that the gravitational time delay between the
\spockone location and the \spocktwo location is $<$60 days
(Table~\ref{tab:LensModelPredictions}).  This falls far short of the
observed 223 day span between the two events, suggesting that
\spocktwo is not a time-delayed image of the \spockone event.  As
shown in Figure~\ref{fig:SpockDelayPredictions}, \spockone and
\spocktwo are inconsistent with these predicted time delays if one
assumes that they are delayed images of a single event.  However, if
these were independent events, then a time delay on the order of tens
of days between image 11.1 and 11.2 could have resulted in
time-delayed events that were missed by the \HST imaging of this
field.

The models also predict absolute magnification values between about
$\mu=10$ and $\mu=200$ for both events. This wide range is due
primarily to the close proximity of the lensing critical curve (the
region of theoretically infinite magnification) for sources at $z=1$.
The lensing configuration consistently adopted for this cluster
assumes that the arc comprises two mirror images of the host galaxy
(labeled 11.1 and 11.2 in
Figure~\ref{fig:SpockDetectionImages})\cite{Zitrin:2013a, Jauzac:2014,
  Johnson:2014, Richard:2014, Diego:2015a, Grillo:2015, Hoag:2016,
  Sebesta:2016, Caminha:2017}.  This implies that a single critical
curve passes roughly midway between the two \spock locations.  The
location of the critical curve varies significantly among the models
(Figure~\ref{fig:SpockCriticalCurves}), and is sensitive to many
parameters that are poorly constrained. We find that it is possible to
make reasonable adjustments to the lens model parameters so that the
critical curve does not bisect the \spock host arc, but instead
intersects both of the \spock locations (see Supplementary
Note~\ref{sec:LensModelVariations}).  Such lensing configurations can
qualitatively reproduce the observed morphology of the \spock host
galaxy, but they are disfavored by a purely quantitative assessment of
the positional strong-lensing constraints.

\subsection{Ruling Out Common Astrophysical Transients.}

There are several categories of astrophysical transients that can be
rejected based solely on characteristics of the \spockone and
\spocktwo light curves, shown in Figure~\ref{fig:LightCurves}. Neither
of the \spock events is {\it periodic}, as expected for stellar
pulsations such as Cepheids, RR Lyrae, or Mira variables. Stellar
flares can produce rapid optical transient phenomena, but the total
energy released by even the most extreme stellar
flare\cite{Karoff:2016} falls far short of the observed energy release
from the \spock transients. We can also rule out active galactic
nuclei (AGN), which are disfavored by the quiescence of the \spock
sources between the two observed episodes and the absence of any of
the broad emission lines that are often observed in AGN.
Additionally, no x-ray emitting point source was detected in 7 epochs
from 2009 to 2014, including \Chandra X-ray Space Telescope imaging
that was coeval with the peak of infrared emission from \spocktwo.

Many types of stellar explosions can generate isolated transient
events, and a useful starting point for classification of such objects
is to examine their position in the phase space of peak luminosity
(\Lpk) versus decline time\cite{Kulkarni:2007}.
Figure~\ref{fig:PeakLuminosityDeclineTime} shows our two-dimensional
constraints on \Lpk and the decline timescale \t2 (the time over which
the transient declines by 2 mag) for the \spock events,
accounting for the range of lensing magnifications ($10<\mu<200$)
derived from the cluster lens models.  The \spockone and \spocktwo
events are largely consistent with each other, and if both events are
representative of a single system (or a homogeneous class) then the
most likely peak luminosity and decline time (the region with the most
overlap) would be $L_{\rm pk}\approx10^{41}$ erg s$^{-1}$ and $t_2\approx1$
day.

The relatively low peak luminosities and the very rapid rise and fall
of both \spock light curves are incompatible with all categories of
stellar explosions for which a significant sample of observed events
exists.  This includes the common Type Ia SNe and core-collapse SNe,
as well as the less well-understood classes of superluminous
SNe\cite{Gal-Yam:2012}, Type Iax SNe\citep{Foley:2013a}, fast optical
transients\cite{Drout:2014}, Ca-rich SNe\cite{Kasliwal:2012}, and
luminous red novae\cite{Kulkarni:2007}.

The SN-like transients that come closest to matching the observed
light curves of the two \spock events are the ``kilonova'' class and
the ``.Ia'' class.  Kilonovae are a category of
optical/near-infrared transients that may be generated by the merger
of a neutron star (NS) binary\cite{Li:1998,Tanvir:2013}.  The .Ia class is
produced by He shell explosions that are expected to arise from AM
Canum Venaticorum (AM CVn) binary star systems undergoing He mass
transfer onto a white dwarf primary star\cite{Bildsten:2007}.  The
\spock light curves exhibited a slower rise time than is expected for
a kilonova event\cite{Barnes:2013, Kasen:2015}, and a
faster decline time than is anticipated for a .Ia
event\cite{Shen:2010}.

Another problem for all of these catastrophic stellar explosion models
is that they cannot explain the appearance of {\it repeated} transient
events.  The kilonova progenitor systems are completely disrupted at
explosion, as is the case for all normal SN explosions.  For .Ia
events, even if an AM CVn system could produce repeated He shell
flashes of similar luminosity, the period of recurrence would be
$\sim10^5$ yr, making these effectively non-recurrent sources.
%Thus, to reconcile any such cataclysmic explosion model with the two
%observed \spock events we would need to invoke a highly serendipitous
%occurrence of two unrelated peculiar explosions in the same host
%galaxy in the same year.

Although the two events were most likely not {\it temporally}
coincident, all of our lens models indicate that it is entirely
plausible for the two \spock events to be {\it spatially} coincident:
a single location at the source plane can be mapped to both \spock
locations to within the positional accuracy of the model
reconstructions ($\sim$0.6\arcsec in the lens plane). This is
supported by the fact that the host-galaxy colors and spectral indices
at each \spock location are indistinguishable within the uncertainties
(see Supplementary Figure~\ref{fig:HostProperties} and Supplementary
Table~\ref{tab:HostProperties}).  Thus, to accommodate all of the
observations of the \spock events with a single astrophysical source,
we turn to two categories of stellar explosion that are sporadically
recurrent: luminous blue variables (LBVs) and recurrent novae (RNe).

\subsection{Luminous Blue Variable.}

The transient sources categorized as LBVs are the result of eruptions
or explosive episodes from massive stars ($>10$ \Msun).
%\footnote{We use
%  the term LBV to encompass any massive stars producing sporadic
%  bright optical transient events. Such
%  transients are variously labeled LBVs, ``SN impostors,'' or SN
%  2008S-like transients.}
The class is exemplified by examples such as P Cygni, $\eta$ Carinae
(\etaCar), and S Doradus\cite{Smith:2011b, Kochanek:2012}.  Although
most giant LBV eruptions have been observed to last much longer than
the \spock events\cite{Smith:2011b}, some LBVs have exhibited repeated
rapid outbursts that are broadly consistent with the very fast \spock
light curves (see Supplementary
Figure~\ref{fig:LBVLightCurveComparison}). Because of this common
stochastic variability, the LBV hypothesis does not have any trouble
accounting for the \spock events as two separate episodes.

Two well-studied LBVs that provide a plausible match to the observed
\spock events are ``SN 2009ip''\cite{Maza:2009} and
NGC3432-LBV1\cite{Pastorello:2010}.  Both exhibited multiple brief
transient episodes over a span of months to years\cite{Miller:2009,
  Li:2009, Berger:2009, Pastorello:2010}.  Unfortunately,
for these outbursts we have only upper limits on the decline
timescale, $t_2$, owing to the relatively sparse photometric sampling.
Recent studies have shown that SN 2009ip-like LBV transients have remarkably
similar light curves, leading up to a final terminal SN
explosion\cite{Kilpatrick:2017, Pastorello:2017}.
Figure~\ref{fig:PeakLuminosityDeclineTime}b shows that both \spock
events are consistent with the observed luminosities and decline times
of these fast and bright LBV outbursts -- though the \spock events
would be among the most rapid and most luminous LBV eruptions ever
seen.

In addition to those relatively short and very bright giant eruptions,
most LBVs also commonly exhibit a slower underlying variability. P
Cygni and \etaCar, for example, slowly rose and fell in brightness by
$\sim$1 to 2 mag over a timespan of several years before and after
their historic giant eruptions.  Such variation has not been detected
at the \spock locations. Nevertheless, given the broad
range of light-curve behaviors seen in LBV events, we cannot reject
this class as a possible explanation for the \spock system.

The total radiated energy of the \spock events is in the range
$10^{44}<E_{\rm rad}<10^{47}$ erg (see Methods), which falls well
within the range of plausible values for a major LBV outburst.  From
this measurement we can derive constraints on the luminosity of the
progenitor star, by assuming that the energy released is generated
slowly in the stellar interior and is in some way ``bottled up'' by
the stellar envelope, before being released in a rapid mass ejection
(see Methods).  With this approach we a quiescent luminosity of
$L_{\rm qui}\approx10^{39.5}$~erg~s$^{-1}$ ($M_V\approx-10$ mag).  This value is
fully consistent with the expected range for LBV progenitor stars
(e.g., \etacar has $M_V\approx-12$ mag and the faintest known LBV progenitors
such as SN 2010dn have $M_V\approx-6$ mag).


\subsection{Recurrent Nova.}\label{sec:RNe}

Novae occur in binary systems in which a white dwarf star accretes
matter from a less massive companion, leading to a burst of nuclear
fusion in the accreted surface layer that causes the white dwarf to
brighten by several orders of magnitude, but does not completely
disrupt the star. The mass transfer from the companion to the white
dwarf may restart after the explosion, so the cycle may begin again
and repeat after a period of months or years.  When this recurrence
cycle is directly observed, the object is classified as a recurrent
nova (RN).

The light curves of many RN systems in the Milky Way are similar in
shape to the \spock episodes, exhibiting a sharp rise ($<10$ days in
the rest-frame) and a similarly rapid decline (see Supplementary
Information and Supplementary
Figure~\ref{fig:RecurrentNovaLightCurveComparison}).  This is
reflected in Figure~\ref{fig:PeakLuminosityDeclineTime}, where novae
are represented by a grey band that traces the empirical constraints
on the maximum magnitude vs.\ rate of decline (MMRD) relation for
classical novae\cite{DellaValle:1995, Downes:2000, Shafter:2011,
  Kasliwal:2011a}.

The RN model can provide a natural explanation for having two separate
explosions that are coincident in space but not in time.  However, the
recurrence timescale for \spock in the rest frame is $120\pm30$ days,
which would be a singularly rapid recurrence period for a RN system.
The RNe in our own Galaxy have recurrence timescales of 10--98
years\cite{Schaefer:2010}.  The fastest measured recurrence timescale
belongs to M31N 2008-12a, which has exhibited a new outburst every
year from 2008 through 2016\cite{Tang:2014, Darnley:2014,
  Darnley:2015, Henze:2015, Darnley:2016}.  Although this
M31 record-holder demonstrates that very rapid recurrence is possible,
classifying \spock as a RN would still require a very extreme
mass-transfer rate to accommodate the $<1$ year recurrence.

Another major concern with the RN hypothesis is that the two \spock
events are substantially brighter than all known novae---perhaps by as
much as 2 orders of magnitude.  This is exacerbated by the
observational and theoretical evidence indicating that
rapid-recurrence novae have less energetic eruptions\cite{Yaron:2005}
(see Supplementary Information and Supplementary Figure
\ref{fig:RecurrentNovaRecurrenceComparison}).
%One might
%attempt to reconcile the \spock luminosity more comfortably with the
%nova class by assuming a significant lensing magnification for one of
%the two events. This would drive down the intrinsic luminosity,
%perhaps to $\sim10^{40}$ erg s$^{-1}$, on the edge of the nova region.
%However, if there is only a single critical curve subtending the
%\spock host galaxy arc then pushing that critical curve to be closer
%to one of the \spock events would pull the critical curve away from
%the other \spock position. This would make that second event {\it more
%  inconsistent} with observed nova peak luminosities.
%The combination of a rapid recurrence timescale and unusually high
%peak luminosity for the \spock events is at odds with theoretical
%expectations and empirical constraints for RNe.
Although the RN model
is not strictly ruled out, we can deduce that if the \spock transients
are caused by a single RN system, then that progenitor system would be
among the most extreme white dwarf binary systems yet known.

% As discussed above, modified lens models may allow for two critical
% curves crossing the host arc close to the \spock events
% (Methods~\ref{sec:Microlensing}).  This would make the RN hypothesis
% more plausible as the inferred lensing magnification would
% simultaneously increase for both \spockone and \spocktwo, thereby
% bringing the intrinsic luminosity of both events into line with the
% expectations for a rapidly recurrent nova. This would, however,
% require substantial fine-tuning of the lens models in order to
% accommodate multiple critical curves that still map the spock events
% back to the same location on the source plane.



\subsection{Microlensing.}\label{sec:MicroLensing}

In the presence of strong gravitational lensing it is possible to
generate a transient event from lensing effects alone.  In this case
the background source has a steady luminosity but the relative motion
of the source, lens, and observer causes the magnification of that
source (and therefore the apparent brightness) to change rapidly with
time.  An isolated strong lensing event with a rapid timescale can be
generated when a background star crosses over a lensing caustic (the
mapping of the critical curve back on to the source plane).  In the
case of a star crossing the caustic of a smooth lensing potential, the
amplification of the source flux would increase (decrease) with a
characteristic $t^{-1/2}$ profile as it moves toward (away from) the
caustic. This slowly evolving light curve then transitions to a very
sharp decline (rise) when the star has moved to the other side of the
caustic\cite{Schneider:1986, MiraldaEscude:1991}.  With a more complex
lens comprising many compact objects, the light curve would exhibit a
superposition of many such sharp peaks\cite{Lewis:1993, Diego:2017}.

The peculiar transient MACS J1149 LS1, observed behind the Hubble
Frontier Fields cluster MACS J1149.6+2223, has been proposed as the
first observed example of such a stellar caustic crossing
event\cite{Kelly:2017}. Such events may be expected to appear more
frequently in strongly lensed galaxies that have small angular
separation from the center of a massive cluster. In such a situation,
our line of sight to the lensed background galaxy passes through a
dense web of overlapping microlenses caused by the intracluster stars
distributed around the center of the cluster. This has the effect of
``blurring'' the magnification profile across the cluster critical
curve, making it more likely that a single (and rare) massive star in
the background galaxy gets magnified by the required factor of
$\sim10^5$ to become visible as a transient caustic-crossing event.
On this basis the \spock host-galaxy images are suitably positioned
for caustic-crossing transients, as they are seen through a relatively
high density of intracluster stars (see Methods)---comparable to that
observed for the MACS J1149 LS1 transient.

The characteristic timescale of a canonical caustic-crossing event
would be on the order of hours or days (see Supplementary
Information), which is comparable to the timescales observed for the
\spock events. Gravitational lensing is achromatic as long as the size
of the source is consistent across the spectral energy distribution
(SED).  This means that the color of a caustic-crossing transient will
be roughly constant.  Using simplistic linear interpolations of the
observed light curves (see Methods), we find that the inferred color
curves for both \spock events are marginally consistent with this
expectation of an unchanging color (Supplementary
Figure~\ref{fig:ColorCurves}).

In the baseline lensing configuration adopted above---where a single
critical curve subtends the \spock host galaxy arc---these events
cannot plausibly be explained as stellar caustic crossings, because
neither transient is close enough to the single critical curve to
reach the required magnifications of $\mu\approx10^6$.  Some of our lens
models can, however, be modified so that instead of just two host
images, the lensed galaxy arc is made up of many more images of the
host, with multiple critical curves subtending the arc where the
\spock events appeared (Figure~\ref{fig:SpockCriticalCurves}).
If this alternative lensing
situation is correct, then similar microlensing transients would be
expected to appear at different locations along the host-galaxy arc,
instigated by new caustic-crossing episodes from different stars in
the host galaxy.

\subsection{The Rate of Similar Transients.}\label{sec:Rates}

Although we lack a definitive classification for these events, we can
derive a simplistic estimate of the rate of \spock-like transients by
counting the number of strongly lensed galaxies in the HFF clusters
that have sufficiently high magnification that a source with
$M_{V}=-14$ mag would be detected in \HST imaging. There are only six
galaxies that satisfy that criteria, all with $0.5<z<1.5$
(Methods).  Each galaxy was observed by the high-cadence HFF program
for an average of 80 days.  Treating \spockone and \spocktwo as
separate events leads to a very rough rate estimate of 1.5 \spock-like
events per galaxy per year.

Derivation of a volumetric rate for such events would require a
detailed analysis of the lensed volume as a function of redshift, and
is beyond the scope of this work. Nevertheless, a comparison to rates
of similar transients in the local universe can inform our assessment
of the likelihood that the \spock events are unrelated.  A study of
very fast optical transients with the Pan-STARRS1 survey derived a
rate limit of $\lesssim0.05$ Mpc$^{-3}$ yr$^{-1}$ for transients
reaching $M\approx -14$ mag on a timescale of $\sim$1
day\citet{Berger:2013b}.  This limit, though several orders of
magnitude higher than the constraints on novae or SNe, is sufficient
to make it exceedingly unlikely that two unrelated fast optical
transients would appear in the same galaxy in a single year.
Furthermore, we have observed no other transient events with similar
luminosities and light curve shapes in high-cadence surveys of five
other Frontier Fields clusters. Indeed, all other transients detected
in the primary HFF survey have been fully consistent with normal SNe.
Thus, we have no evidence to suggest that transients of this kind are
common enough to be observed twice in a single galaxy in a single
year.
