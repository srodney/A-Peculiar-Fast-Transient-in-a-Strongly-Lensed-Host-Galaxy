\section{Summary and Conclusions}
\label{sec:Summary}

The key observed properties of the two \spock events are as follows:

\begin{enumerate}
\item{\label{obs:twoevents} Two events appeared in separate images of a single galaxy,
  separated by $\sim$120 days in the rest frame (240 days in the
  observer frame). \label{itm:TwoEvents}}
\item{\label{obs:timedelay} Lensing models predict a time delay between the two host images
  of $\pm60$ days or less in the observer frame.}
\item{\label{obs:timescale} Each event lasted for $<$15 rest-frame days\label{itm:FastLC}}
\item{\label{obs:luminosity} After correcting for lensing magnification of $\mu\sim30$, both
  events reach a peak luminosity of $\sim10^{41}$ erg s$^{-1}$
  \textcolor{red}{(NEED TO REFINE THIS NUMBER)}}
\item{\label{obs:xray} \textcolor{red}{ADD IN X-RAY AND GAMMA RAY CONSTRAINTS}}
\end{enumerate}


To explain the observation of two transient episodes in different host
galaxy images (observation \ref{obs:twoevents}), there are four possible
scenarios:

\begin{enumerate}[(A)]
\item{\label{case:microlensing} An intrinsically static source that
  appears transient due to time-varying magnification (e.g. a
  microlensing event).}
\item{\label{case:timedelay}A single transient source (e.g., a stellar
  explosion) that appeared to us twice with an intervening time delay
  due to strong gravitational lensing.}
\item{\label{case:twoexplosions} Two separate events from distinct
  transient sources (e.g., two explosions from unrelated stars).}
\item{\label{case:recurrent} Two separate events from a single
  astrophysical source (e.g., a recurrent nova).}
\end{enumerate}

