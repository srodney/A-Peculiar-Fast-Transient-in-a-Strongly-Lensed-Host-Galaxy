
\begin{abstract}
A massive galaxy cluster can serve as a magnifying glass for distant
stellar populations, with strong gravitational lensing exposing
details in the lensed background galaxies that would otherwise be
undetectable.  Here we describe two transient events observed in
Hubble Space Telescope imaging of a strongly lensed galaxy at
redshift $z=1.0054\pm0.0002$ behind the \fullmacs0416 cluster.  These
transients---designated \spockone and \spocktwo and collectively
nicknamed ``Spock''---were faster and fainter than any supernova (SN),
but significantly more luminous than a classical nova.  Models of the
cluster lens suggest that these events may have been {\it spatially}
coincident at the source plane, but most likely not {\it temporally}
coincident.  We find that \spock can be explained as a luminous blue
variable, a recurrent nova, or a pair of stellar
microlensing events. A definitive classification will require
further improvements in the lens models, along with
high-cadence monitoring to find new transient episodes.
\end{abstract}
