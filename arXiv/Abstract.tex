
\begin{abstract}
A massive galaxy cluster can serve as a magnifying glass for distant
stellar populations, with strong gravitational lensing exposing details
in the lensed background galaxies that would otherwise be undetectable.
The \fullmacs0416 cluster (hereafter \macs0416) is one of the most
efficient lenses in the sky, and in 2014 it was observed with high-cadence
imaging from the Hubble Space Telescope (\HST). Here we describe two
unusual transient events that appeared behind \macs0416 in a strongly lensed galaxy at redshift
$z=1.0054\pm0.0002$. These transients---designated
\spockone and \spocktwo and collectively nicknamed ``Spock''---were
faster and fainter than any supernova (SN), but significantly more luminous
than a classical nova. They reached peak luminosities of $\sim10^{41}$
erg s$^{-1}$ ($M_{AB}<-14$ mag) in $\lesssim$5 rest-frame days, then faded
below detectability in roughly the same time span.  Models of the
cluster lens suggest that these events may be {\it spatially}
coincident at the source plane, but are most likely not {\it
  temporally} coincident.  We find that \spock can be explained as a
luminous blue variable (LBV), a recurrent nova (RN), or a pair of stellar
microlensing events.  To distinguish between these hypotheses will
require a clarification of the positions of nearby critical curves,
along with high-cadence monitoring of the field that could detect new
transient episodes in the host galaxy.
\end{abstract}
  
  
