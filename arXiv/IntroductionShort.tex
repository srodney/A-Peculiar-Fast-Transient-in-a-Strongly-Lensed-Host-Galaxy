
 When a star explodes or a relativistic jet erupts from near the edge
of a black hole, the event can be visible across many billions of
light-years.  Such extremely luminous astrophysical transients as
supernovae (SNe), gamma-ray bursts, and quasars are powerful tools for
probing cosmic history and sampling the matter and energy content of
the universe.  Less energetic transients generated by the tumultuous
atmospheres of massive stars or the interactions of close stellar
binaries are also very valuable for understanding stellar evolution
and the physical processes that lead to stellar explosions.  However,
the lower luminosity of such events makes them accessible only in
the local universe, and consequently our census of peculiar transients
at the stellar scale is still highly incomplete.

%Until recently, most surveys searching for extragalactic optical
%transients have been optimized for the discovery of SNe, and
%particularly for Type Ia SNe, because of their value as cosmological
%probes\cite{Weinberg:2013}.  These surveys have favored a cadence of
%several days between return visits, with relatively short exposures to
%maximize the area of sky covered while remaining sensitive to their
%primary targets---relatively bright Type Ia SNe.

Although recent surveys are beginning to discover progressively more 
categories of rapidly changing optical
transients\cite{Kasliwal:2011a,Drout:2014}, most programs remain
largely insensitive to transients with peak brightness and timescales
comparable to the \spock events\cite{Berger:2013b}.  Future wide-field
observatories such as the Large Synoptic Survey
Telescope\cite{Tyson:2002} will be much more efficient at discovering
such transients, and can be expected to reveal many new categories of
astrophysical transients.

As shown in Figure~\ref{fig:SpockDetectionImages}, the \spock events
appeared in \HST imaging collected as part of
the Hubble Frontier Fields (HFF) survey\cite{Lotz:2017}, a multi-cycle
program for deep imaging of 6 massive galaxy clusters and associated
``blank sky'' fields observed in parallel.  \HST is not an efficient
wide-field survey telescope, and the HFF survey was not designed with
the discovery of peculiar extragalactic transients as a core
objective.  However, the HFF program has unintentionally opened an
effective window of discovery for such events.  Very faint sources at
relatively high redshift ($z\gtrsim1$) in these fields are made
detectable by the substantial gravitational lensing magnification from
the foreground galaxy clusters.  Very rapidly evolving sources are
also more likely to be found, owing to the necessity of a rapid cadence
for repeat imaging in the HFF program.

