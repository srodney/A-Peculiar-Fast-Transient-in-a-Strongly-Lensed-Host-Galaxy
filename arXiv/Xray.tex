\subsection{X-ray Nondetections.}\label{sec:Xray}

The \MACS0416 field was observed by the \Swift X-Ray Telescope
and UltraViolet/Optical Telescope in April 2013.  No source was
detected near the locations of the \spock events (N. Gehrels, private
communication).  The field was also observed by \Chandra with the
ACIS-I instrument for three separate programs.  On June 7, 2009 it was
observed for GO program 10800770 (PI: H.\,Ebeling).  It was revisited
for GTO program 15800052 (PI: S.\,Murray) on November 20, 2013 and for
GO program 15800858 (PI: C.\, Jones) on June 9, August 31, November
26, and December 17, 2014. These \Chandra images show no evidence for
an x-ray emitting point source near the \spock locations on those
dates (S. Murray, private communication).

The \Chandra observations that were closest in time to the observed
\spock events were those taken in August and November, 2014.  The
August 31 observations were coincident with the observed peak of
rest-frame optical emission for the \spocktwo event (on MJD
56900). The November 26 observations correspond to 44 rest-frame days
after the peak of the \spocktwo event. If the \spock events are
UV/optical nova eruption, then these observations most likely did not
coincide with the nova system's supersoft x-ray phase. For a RN system
the x-ray phase typically initiates after a short delay, and persists
for a span of only a few weeks. For example, the most rapid recurrence
nova known, M31N 2008-12a, has exhibited a supersoft x-ray phase from
6 to 18 days after the peak of the optical emission
\citep{Henze:2015a}.
