Simplistic linear fits to the \spock light curves, used to measure the rise time and decay time of the two events.  The \spockone light curve is shown in the top panel, and \spocktwo in the bottom. Filled points with error bars plot the observed brightness of each event in AB magnitudes as a function of rest-frame time (for $z=1.0054$). Piece-wise linear fits are shown for the four bands that have enough points for fitting: in the top panel fits are plotted for the F814W band (solid green lines) and the F435W band (dashed cyan lines), while in the bottom panel fits are shown for F160W (solid maroon) and F125W (dashed scarlet).  Open diamonds in each panel show three examples of assumptions for the time of peak brightness, $t_{\rm pk}$ (i.e. the position where the rising piece of the linear fit ends).  Open circles mark the corresponding point, $t_{\rm pk} + t_3$, at which the fading transient would have declined in brightness by 3 magnitudes.  See text for details on the fitting procedure.
\label{fig:LinearLightCurveFits}