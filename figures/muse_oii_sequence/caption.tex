\label{fig:MUSEOIISequence}
Measurements of the \forbidden{O}{ii} $\lambda\lambda$3726,3729 doublet, observed
with MUSE after both \spock events had faded.  The upper left panel
shows the 12 apertures with radius 0.6\arcsec that were used for the
extractions reported in Table~\ref{tab:MuseLineFits}.  Odd-numbered
apertures are plotted with solid lines, while even-numbered apertures
are shown as unlabeled dashed circles.  The apertures centered on
the \spock-NW and SE locations are highlighted in orange and magenta,
respectively.  The 1-D spectra extracted from these \spock locations
are shown in the upper right and lower right panels, centered around
the observed wavelength of the \forbidden{O}{ii} doublet, and
normalized to reach a value of unity at the peak of the $\lambda$3729
line.  Dashed vertical lines mark the vacuum wavelengths of the
doublet, redshifted to $z=1.0054$. The width of the shaded region
indicates the $1\sigma$ uncertainty in the measured flux.  Below each
spectrum, a residual plot shows the flux that remains after
subtracting off a mean spectrum constructed from the normalized
spectra of the odd-numbered apertures.  The lower left panel shows the
same residual spectra constructed for the odd-numbered apertures,
demonstrating that the \forbidden{O}{ii} line profile does not exhibit
any significant gradients across the length of the host galaxy arc.
