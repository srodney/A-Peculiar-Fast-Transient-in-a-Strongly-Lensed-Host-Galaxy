\label{fig:PeakLuminosityDeclineTime}
Peak luminosity vs. decline time for \spock and assorted categories of
explosive transients.  Observed constraints of the \spock events are
plotted as overlapping colored bands, along the left side of the
figure.  \spockone is shown as cyan and blue bands, corresponding to
independent constraints drawn from the F435W and F814W light curves,
respectively.  For \spocktwo the scarlet and maroon bands show
constraints from the F125W and F160W light curves, respectively.  The
width and height of these bands incorporates the uncertainty due to
magnification (we adopt $7<\mu_{\rm NW}<485$ and $7<\mu_{\rm SE}<185$; see Table ~\ref{tab:LensModelPredictions}) and the time of peak.  In
the top panel, ellipses and rectangles mark the luminosity and
decline-time regions occupied by various explosive transient classes.
Filled shapes show the empirical bounds for transients with a
substantial sample of known events. Dashed regions mark theoretical
expectations for rare transients that lack a significant sample size:
the ``.Ia'' class of white dwarf He shell detonations and the kilonova
class from neutron star mergers.  Grey bands in both panels show the
MMRD relation for classical novae.  In the lower panel, circles mark
the observed peak luminosities and decline times for classical novae,
while black `+' symbols mark recurrent novae from our own galaxy.  The
large cross labeled at the bottom shows the rapid recurrence nova M31N
2008-12a.  Each orange diamond marks a separate short transient event
from the two rapid LBV outburst systems, SN
2009ip\cite{Pastorello:2013} and NGC3432-LBV1 (a.k.a. SN
2000ch)\cite{Pastorello:2010}.  These LBV events provide only upper
limits on the decline time due to limited photometric sampling.
