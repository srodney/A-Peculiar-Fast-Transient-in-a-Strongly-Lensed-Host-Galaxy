\label{fig:PeakLuminosityDeclineTimeWide}
Peak luminosity vs. decline time for \spock and other astrophysical
transients.  Observed constraints of the \spock events are plotted as
overlapping colored bands, along the left side of the
figure.  \spockone is shown as cyan and blue bands, corresponding to
independent constraints drawn from the F435W and F814W light curves,
respectively.  For \spocktwo the scarlet and maroon bands show
constraints from the F125W and F160W light curves, respectively.  The
width and height of these bands incorporates the uncertainty due to
magnification (we adopt $10<\mu<100$) and the time of peak (using
linear fits as shown in Figure~\ref{fig:LinearLightCurveFits}).
Ellipses and rectangles mark the luminosity and decline-time regions
occupied by other transient classes.  Filled shapes show the empirical
bounds for transients with a substantial sample of known
events. Dashed regions mark theoretical expectations for rare
transients that lack a significant sample size: the ``.Ia'' class of
white dwarf He shell detonations \citep{Bildsten:2007,Shen:2010} and
the kilonova class from neutron star
mergers \citep{Li:1998,Kulkarni:2005,Kasen:2015}.
