%%%%%%%%%%%%%%%%%%%%%%%%%%%%%%%%%%%%%%%%%%%%%%%%%%%%%%%%%%%%
% MAIN DOCUMENT TEXT
%%%%%%%%%%%%%%%%%%%%%%%%%%%%%%%%%%%%%%%%%%%%%%%%%%%%%%%%%%%%

\shorttitle{A Type Ia SN Behind Abell 2744}
\shortauthors{Rodney et al.}

\begin{document}

\title{Illuminating a Dark Lens : A Type Ia Supernova Magnified by \\ the Frontier Fields Galaxy Cluster Abell 2744}

 \author{S.~A.~Rodney\altaffilmark{\affilref{USC},}}
 \affilreftxt{USC}{\USC}
 \email{srodney@sc.edu}

 \author{The FrontierSN Team}


\begin{abstract}
{
SN \tomas\ is a Type Ia Supernova (SN) discovered at
$z=1.3457\pm0.0001$ behind the galaxy cluster Abell 2744 ($z=0.308$).
In a cosmology-independent analysis, we find that \tomas\ is
$0.77\pm0.15$ magnitudes brighter than unlensed Type Ia SNe at similar
redshift, implying a lensing magnification of $\mu_{\rm
obs}=2.03\pm0.29$.  This observed magnification provides a rare
opportunity for a direct empirical test of galaxy cluster lens models.
Here we test 17 lens models, 13 of which were generated before the SN
magnification was known, qualifying as pure ``blind tests''.  The
models are collectively fairly accurate: 8 of the models deliver
median magnifications that are consistent with the measured $\mu$ to
within 1$\sigma$.  However, there is a subtle systematic bias: the
significant disagreements all involve models {\it overpredicting} the
magnification. We evaluate possible causes for this mild bias, and
find no single physical or methodological explanation to account for
it.  We do find that model accuracy can be improved to some extent
with stringent quality cuts on multiply-imaged systems, such as
requiring that a large fraction have spectroscopic redshifts.  In
addition to testing model accuracies as we have done here, Type Ia SN
magnifications could also be used as inputs for future lens models of
Abell 2744 and other clusters, providing valuable constraints in
regions where traditional strong- and weak-lensing information is
unavailable.}
\end{abstract}

\keywords{ supernovae: general, supernovae: individual: HFF14Tom, 
galaxies: clusters: general, galaxies: clusters: individual: Abell 2744, 
gravitational lensing: strong, gravitational lensing: weak  }

\section{Introduction}
\label{sec:Introduction}

Galaxy clusters can be used as cosmic telescopes to magnify distant
background objects through gravitational lensing, which can
substantially increase the reach of deep imaging surveys.  The lensing
magnification enables the study of objects that would otherwise be
unobservable because they are either intrinsically
faint \citep[e.g.][]{Schenker:2012,Alavi:2014} or extremely
distant \citep[e.g.][]{Franx:1997,Ellis:2001,Hu:2002,Kneib:2004,Richard:2006,Richard:2008,Bouwens:2009a,Maizy:2010,Zheng:2012,Coe:2013,Bouwens:2014,Zitrin:2014b}.
Background galaxies are also {\it spatially} magnified, allowing for
studies of the internal structure of galaxies in the early universe
with resolutions of $\sim$100
pc \citep[e.g.][]{Stark:2008,Jones:2010,Yuan:2011,Wuyts:2014,Livermore:2015}.

